%-----------------------------------------------------------------------------%
\chapter*{\Judul}
%-----------------------------------------------------------------------------%
\singlespacing
\begin{center}
    
    \vspace{-4em}
    
    \penulis
    
	\bigskip
    
    \textbf{ABSTRAK}
    
\end{center}

% \chapter*{Abstrak}

\vspace*{0.2cm}
{
	\setlength{\parindent}{0pt}

	\bigskip
	\bigskip

Perkembangan teknologi memberikan dampak yang signifikan di berbagai bidang, termasuk dalam dunia perkantoran. Salah satu penerapannya pada bidang sumber daya manusia adalah \textit{Human Resource Information System} (HRIS), yang dirancang untuk meningkatkan efisiensi dan efektivitas dalam pengelolaan karyawan. PT Visi Karya Nusantara mengembangkan sistem HRIS untuk mendigitalisasi proses-proses penting, terutama pengelolaan kontrak, komponen gaji, hingga penjadwalan acara. Sistem ini dibangun menggunakan ReactJS TypeScript pada sisi \textit{frontend} serta ExpressJS pada sisi \textit{backend}. Fitur yang dikembangkan mencakup modul \textit{Contract, Payroll System, Attendance}, dan \textit{Event}. HRIS yang telah dibangun kini telah menjadi produk dan berada pada tahap \textit{production}, sehingga dapat digunakan oleh \textit{client} dalam operasional sehari-hari. Ke depannya, sistem ini diharapkan terus berkembang untuk mendukung peningkatan produk serta memenuhi kebutuhan \textit{client} secara berkelanjutan.
	\bigskip
 
% Kata kunci urut abjad
% 3 – 5 kata kunci
	\textbf{Kata kunci:} {\textit{Human Resource Information System, Production, ReactJS TypeScript} } 

\onehalfspacing