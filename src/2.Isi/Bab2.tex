%-----------------------------------------------------------------------------%
\chapter{\babDua}
%-----------------------------------------------------------------------------%

%-----------------------------------------------------------------------------%
\section{Sejarah Singkat Perusahaan}
%-----------------------------------------------------------------------------%
\hspace{2em} PT Visi Karya Nusantara, yang lebih dikenal dengan AfterSix, didirikan pada tahun 2025 dan bergerak di bidang \textit{software house}. Perusahaan ini berlokasi di Start Space Coworking Space Gading Serpong. Saat ini, perusahaan tengah mengerjakan berbagai proyek di antaranya Nine to Six, AKS, IkoVisual, Beeliv, dan KAIYA. Salah satu identitas perusahaan dapat dilihat pada Gambar 2.1, yang menampilkan logo resmi PT Visi Karya Nusantara.

\begin{figure}
    \centering
    \fbox{\includegraphics[width=0.4\linewidth]{assets/pics/Frame 5490.png}}
    \caption{Logo perusahaan PT Visi Karya Nusantara}
      {\small Sumber: \cite{PTVisiKaryaNusantara}}
    \label{fig:logo}
\end{figure}


%-----------------------------------------------------------------------------%
\section{Visi dan Misi Perusahaan}
%-----------------------------------------------------------------------------%
\hspace{2em} Terdapat visi dan misi dari PT Visi Karya Nusantara, yaitu \cite{PTVisiKaryaNusantara}: 
\subsection{Visi}
\hspace{2em} Menjadi perusahaan konsultan teknologi informasi yang memberikan solusi yang optimal, efektif, dan efisien, serta membangun budaya perusahaan yang berintegritas, profesional, dan berkelanjutan.

\subsection{Misi}

\begin{enumerate}
    \item Menyediakan layanan konsultasi teknologi informasi yang berorientasi pada efektivitas dan efisiensi, guna memastikan setiap solusi yang diberikan selaras dengan kebutuhan dan tujuan klien.

    \item Membangun organisasi yang profesional dan berintegritas, melalui pengembangan sumber daya manusia yang adaptif, kompeten, dan berdaya saing, serta menumbuhkan budaya kerja yang kolaboratif, inovatif, dan berkelanjutan.

    \item Berperan aktif dalam mendukung transformasi digital di Indonesia, dengan mengedepankan praktik konsultasi yang beretika, berkualitas, dan berorientasi pada hasil jangka panjang.
\end{enumerate}

%-----------------------------------------------------------------------------%
\section{Struktur Organisasi Perusahaan}
%-----------------------------------------------------------------------------%

\hspace{2em} Struktur organisasi di PT Visi Karya Nusantara dapat dilihat pada Gambar 2.2, yang menggambarkan hubungan antar posisi serta alur koordinasi dalam perusahaan \cite{PTVisiKaryaNusantara}.

\begin{figure}
    \centering
     \fbox{\includegraphics[width=1 \linewidth]{assets/pics/stucture.jpg}}
   \caption{Struktur organisasi pada PT Visi Karya Nusantara}
   {\small Sumber: \cite{PTVisiKaryaNusantara}}
  \label{fig:enter-label}
\end{figure}

\pagebreak
\hspace{2em} Struktur organisasi tersebut terdiri dari beberapa peran utama yang memiliki tanggung jawab masing-masing dalam mendukung operasional perusahaan. Pada tingkat tertinggi, \textit{Director} bertanggung jawab atas pengambilan keputusan strategis serta memastikan seluruh kegiatan perusahaan berjalan sesuai visi dan misi yang telah ditetapkan. Di bawahnya terdapat \textit{Software Engineer Manager} yang berperan dalam mengelola tim teknis, memantau produktivitas, serta mengoordinasikan alur kerja antara \textit{Software Engineer} dan \textit{UI/UX Designer} agar setiap proyek dapat diselesaikan secara efektif dan tepat waktu.

\hspace{2em} Selanjutnya, \textit{Software Engineer} memiliki tanggung jawab dalam merancang, mengembangkan, dan memelihara sistem perangkat lunak sehingga fungsionalitas sistem tetap optimal dan sesuai kebutuhan perusahaan. Pada sisi lain, \textit{UI/UX Designer} bertugas merancang tampilan dan pengalaman pengguna yang intuitif, menarik, dan mudah digunakan untuk memastikan kenyamanan pengguna dalam berinteraksi dengan sistem. Setiap peran dalam struktur organisasi ini saling berkoordinasi dan bekerja sama untuk mencapai tujuan perusahaan secara optimal serta mendukung pengembangan produk yang berkualitas.


