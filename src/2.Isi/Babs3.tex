%-----------------------------------------------------------------------------%
\chapter{\babTiga}


%-----------------------------------------------------------------------------%

%-----------------------------------------------------------------------------%
\section{Kedudukan dan Koordinasi}
%-----------------------------------------------------------------------------%
\hspace{2em} Selama pelaksanaan kegiatan magang di PT Visi Karya Nusantara, posisi yang ditempati adalah Software Engineer di bawah pengawasan dan bimbingan Atanasius Raditya Herkristito. Tugas utama selama magang adalah mengembangkan \textit{Human Resource Information System} (HRIS), yang mencakup perancangan dan implementasi antarmuka untuk meningkatkan efisiensi sistem dalam pengelolaan data kepegawaian.

\begin{figure}
    \centering
     \fbox{\includegraphics[width=0.6 \linewidth]{assets/pics/stucture internal.jpg}}
   \caption{Struktur organisasi Pengembang}
  \label{fig:enter-label}
\end{figure}

\hspace{2em} Dalam pengembangan HRIS, tim terdiri dari satu \textit{Software Engineer Manager}, satu \textit{UI/UX Designer}, dan empat \textit{Software Engineer}. Selama pengembangan, komunikasi dan koordinasi dilakukan melalui aplikasi Discord. Tinjauan progres dan \textit{code review} dilakukan setiap minggu, tepatnya di hari Kamis. Selain itu, terdapat sesi evaluasi dengan seluruh tim untuk memberikan \textit{feedback} terkait perkembangan proyek, guna meningkatkan kualitas berikutnya.

%-----------------------------------------------------------------------------%
\section{Tugas yang Dilakukan}

\hspace{2em} Selama Pelaksanaan magang di PT Visi Karya Nusantara, tugas utama yang dikerjakan adalah mengembangkan \textit{Human Resource Information System} (HRIS), yang direncanakan menjadi produk resmi dan digunakan oleh perusahaan. Adapun tugas yang dilakukan meliputi:

\begin{enumerate}
    \item Membangun \textit{frontend} pada modul \textit{contract} dan \textit{payroll system}
    \item Mengembangkan sistem absensi yang mencakup fitur \textit{clock in, clock out}, dan \textit{lunch} pada modul \textit{attendance}.
\item Mengembangkan modul \textit{Event} yang mencakup proses pembuatan kegiatan serta pengumuman (announce) secara langsung pada \textit{dashboard} pengguna.
\item Melakukan manual \textit{testing} terhadap HRIS.
\end{enumerate}


\section{Uraian Pelaksanaan Magang}

\begin{table}
	\centering
	\caption{ Pekerjaan yang dilakukan tiap minggu selama pelaksanaan kerja magang}
	\label{tbl_uraian}
	\begin{tabular}{|c | p{0.75\textwidth}| }
		\hline
		Minggu Ke - & Pekerjaan yang dilakukan \\
		\hline
		1 & \textit{Onboarding} dan mempelajari \textit{flow} sistem HRIS. \\
		\hline
 		2 & Melalukan \textit{research} terkait integrasi modul \textit{Contract} pada HRIS\\
		\hline
 		3 & Mengerjakan UI modul \textit{Contract}. \\
 		\hline
 		4 & Melakukan integrasi API modul \textit{Contract}.\\
 		\hline
 		5 & Melakukan manual \textit{testing} pada modul \textit{Contract}. \\
 		\hline 
 		6 & Mengerjakan UI modul \textit{Payroll System}. \\
 		\hline
            7 & Melakukan integrasi API modul \textit{Payroll System} dan pada \textit{Contract Input}. \\
 		\hline
            8 & Melakukan manual \textit{testing} pada modul \textit{Payroll System}.\\
 		\hline
            9 &  Melakukan perbaikan bug (\textit{bug fixing}) dan proses
        \textit{deployment}.\\
 		\hline
            10 & Melakukan \textit{research} terkait proses \textit{attendance} pada HRIS\\
 		\hline
        11 & Melakukan integrasi API fitur \textit{Attendance}\\
        \hline
        12 & Mengerjakan UI fitur \textit{Lunch Attendance}\\
        \hline
        13 & Melakukan integrasi \textit{Attendance} dengan \textit{Contract Schedule}\\
        \hline
        14 & Melakukan manual \textit{testing} pada fitur \textit{Attendance}\\
        \hline
        15 & Melakukan pengembangan UI modul \textit{Event}\\
        \hline
        16 & Melakukan integrasi API modul \textit{Event}\\
        \hline
        17 & Melakukan manual \textit{testing} pada modul \textit{Event}\\
        \hline
        18 & Melakukan perbaikan bug (\textit{bug fixing}) dan proses \textit{deployment}\\
        \hline
	\end{tabular}
\end{table}

\hspace{2em} Tabel 3.1 di atas merupakan uraian pelaksanaan magang yang dilaksanakan di PT Visi Karya Nusantara. Selama 18 minggu, terlibat dalam pengembangan \textit{frontend} memulai dengan proses \textit{onboarding} dan mempelajari sistem HR. Tahap pertama dimulai dengan pembuatan modul \textit{Contract}, melakukan integrasi API, hingga melakukan manual \textit{testing}. Selanjutnya, pembuatan modul \textit{Payroll System} dan mengintegrasikan dengan modul \textit{Contract}, kemudian melakukan manual \textit{testing}. Berikutnya, mengembangkan sistem \textit{Attendance} yang terintegrasi dengan \textit{Contract}, termasuk penambahan fitur \textit{Lunch Attendance}. Setelah itu, pengembangan modul \textit{Event} dengan penambahan fitur-fitur di dalamnya, serta melakukan integrasi API dan manual \textit{testing}. Lalu, melakukan pengecekan di seluruh halaman dan fitur, melakukan \textit{bug fixing}, serta menjalankan proses \textit{deployment} ke \textit{production} untuk dijadikan produk.

\section{Perancangan}

\hspace{2em} Sub bab ini menjelaskan proses perancangan aplikasi \textit{Human Resource Information System} (HRIS), yang mencakup \textit{User Requirement, Sitemap,} dan \textit{Flowchart}.
\\

\subsection{\textit{User Requirement}}

\hspace{2em} Sebelum memulai pengembangan, \textit{Software Engineer Manager} memberikan arahan terkait tugas yang akan dilakukan dalam pengembangan \textit{Human Resource Information System} (HRIS). Aspek yang perlu dikembangkan di antaranya adalah sebagai berikut:

\begin{enumerate}
    \item Mengembangkan halaman \textit{Contract} untuk pengelolaan kontrak karyawan, meliputi pembuatan, pengeditan, dan penghapusan kontrak. Selain itu, tersedia fitur \textit{approval}, \textit{rejection}, \textit{cancel}, dan \textit{termination} sesuai dengan status kontrak. Karyawan juga dapat melihat detail kontrak mereka melalui halaman \textit{Contract User}.

    \item Mengembangkan halaman \textit{Payroll Setting} untuk pengaturan item gaji dan tunjangan yang digunakan pada \textit{Contract Input}. Halaman ini terdiri dari \textit{Payroll Setting List} yang menampilkan seluruh item payroll yang tersedia, serta \textit{Payroll Setting Input} untuk menambahkan item baru.

    \item Mengembangkan fitur \textit{Attendance System} pada halaman \textit{Dashboard} untuk mengatur proses absensi karyawan. Pembaruan dilakukan dengan menambahkan fitur \textit{clock in} bagi karyawan dengan sistem kerja \textit{shift} maupun \textit{fixed hour}, yang disesuaikan berdasarkan jadwal dan jam kerja pada kontrak masing-masing.

        \item Mengembangkan modul \textit{Event} yang mencakup pembuatan kegiatan dan pengumuman (\textit{announce}) secara langsung pada \textit{dashboard} pengguna. Fitur ini memungkinkan penambahan deskripsi kegiatan, dokumen pendukung, serta pembaruan data secara \textit{real-time} agar pengguna dapat dengan mudah memantau dan mengelola jadwal kegiatan yang akan datang.
    
\end{enumerate}

\subsection{\textit{Sitemap}}

\begin{figure}
    \centering
     \fbox{\includegraphics[width=1 \linewidth]{assets/pics/sitemap A.png}}
  \label{fig:enter-label}
\end{figure}

\begin{figure}
    \centering
     \fbox{\includegraphics[width=1 \linewidth]{assets/pics/sitemap B.png}}
     \caption{\textit{Sitemap Human Resource Infromation System}}
  \label{fig:enter-label}
\end{figure}

\hspace{2em} Gambar 3.2 menunjukkan \textit{sitemap} dari sistem \textit{Human Resource Information System} (HRIS) yang telah dikembangkan. Berdasarkan gambar tersebut, halaman pertama yang akan ditampilkan kepada pengguna adalah halaman \textit{login}. Setelah berhasil masuk, pengguna akan diarahkan ke halaman \textit{dashboard} sebagai tampilan utama. Dari halaman ini, pengguna dapat mengakses berbagai modul yang tersedia, seperti \textit{Contract}, \textit{Payroll Setting}, \textit{Attendance System}, dan halaman-halaman lain yang tersedia berdasarkan akses yang diberikan.

\subsection{\textit{Flowchart}}

\hspace{2em} Pada bagian ini, ditampilkan \textit{flowchart} yang menggambarkan alur kerja sistem modul secara keseluruhan. Alur lengkapnya dapat ditampilkan sebagai berikut: 

\begin{figure}
    \centering
    \fbox{\includegraphics[width=0.7\linewidth]{assets/pics/contract flow.png}}
    \caption{\textit{Flowchart Contract Permission Page}}
    \label{fig:enter-label}
\end{figure}

\hspace{2em} Gambar 3.3 menunjukkan \textit{flowchart} pembagian halaman berdasarkan \textit{permission role} yang diberikan pada halaman \textit{Contract}. Sistem akan melakukan pengecekan terhadap akun yang sedang \textit{login} untuk menentukan \textit{role} yang dimilikinya. Berdasarkan \textit{role} tersebut, sistem akan memberikan izin (\textit{permission}) tertentu yang menentukan halaman atau fitur apa saja yang dapat diakses oleh pengguna di dalam halaman tersebut.


\begin{figure}
    \centering
    \fbox{\includegraphics[width=0.5\linewidth]{assets/pics/contract list.png}}
    \caption{\textit{Flowchart Contract List}}
    \label{fig:enter-label}
\end{figure}

\hspace{2em} Gambar 3.4 memperlihatkan \textit{flowchart} pada halaman \textit{Contract List}. Proses dimulai dengan pemanggilan GET API untuk menampilkan seluruh \textit{contract} yang telah dibuat. Selanjutnya, pengguna yang memiliki \textit{permission create} dapat melakukan pembuatan \textit{contract} baru. Sementara itu, pengguna dengan \textit{permission approval} dapat melakukan persetujuan (\textit{approval}) \textit{contract}.

\begin{figure}
    \centering
    \fbox{\includegraphics[width=0.8\linewidth]{assets/pics/contract create.png}}
    \caption{\textit{Flowchart Contract Create}}
    \label{fig:enter-label}
\end{figure}

\hspace{2em} Gambar 3.5 merupakan \textit{flowchart} pada halaman \textit{Contract Create}. Proses dimulai dengan menampilkan \textit{form field} untuk pengisian data perjanjian karyawan. Pengguna dapat melakukan kustomisasi terhadap data \textit{penalty \& rule}, jadwal kerja karyawan, serta komponen gaji. Setelah seluruh data diisi, sistem akan menampilkan modal konfirmasi terlebih dahulu. Setelah pengguna memberikan konfirmasi, data tersebut akan disimpan ke dalam \textit{database}.

\begin{figure}
    \centering
    \fbox{\includegraphics[width=0.7\linewidth]{assets/pics/contract detail.png}}
    \caption{\textit{Flowchart Contract Detail Approval}}
    \label{fig:enter-label}
\end{figure}

 \hspace{2em} Gambar 3.6 menunjukkan \textit{flowchart} pada halaman \textit{Contract Detail Approval}. Proses dimulai dengan pemanggilan GET API untuk mengambil data kontrak yang dipilih. Jika pengguna memiliki \textit{permission approval}, maka dapat melakukan \textit{accept} atau \textit{reject} terhadap kontrak. Setelah itu, ditampilkan modal konfirmasi, dan kontrak disimpan ke \textit{database} serta dikirim kepada pengguna untuk mengaktifkan masa kontrak.

\begin{figure}
    \centering
    \fbox{\includegraphics[width=0.5\linewidth]{assets/pics/payroll list.png}}
    \caption{\textit{Flowchart Payroll List}}
    \label{fig:enter-label}
\end{figure}

\hspace{2em} Gambar 3.7 adalah \textit{flowchart} pada halaman \textit{Payroll List}. Proses dimulai dengan pemanggilan GET API untuk menampilkan seluruh item \textit{Payroll Setting}. Pengguna dapat membuat item \textit{payroll} baru apabila mendapatkan \textit{permission create}. (flowchartnya ada yang salah)

\begin{figure}
    \centering
    \fbox{\includegraphics[width=0.7\linewidth]{assets/pics/payroll create.png}}
    \caption{\textit{Flowchart Payroll Create}}
    \label{fig:enter-label}
\end{figure}

\hspace{2em} Gambar 3.8 merupakan \textit{flowchart} pada halaman \textit{Payroll Create}. Proses dimulai dengan menampilkan \textit{form field} untuk pengisian \textit{item payroll}. Selanjutnya, akan ditampilkan modal konfirmasi pengajuan. Lalu, data akan disimpan ke dalam \textit{database}.

\begin{figure}
    \centering
    \fbox{\includegraphics[width=0.7\linewidth]{assets/pics/dashboard flow.png}}
    \caption{\textit{Flowchart Attendance System}}
    \label{fig:enter-label}
\end{figure}

\hspace{2em} Gambar 3.9 memperlihatkan \textit{flowchart} pada halaman \textit{Dashboard Permission}. Apabila pengguna merupakan \textit{superadmin}, maka sistem akan menampilkan \textit{dashboard overview} yang berisi data-data internal. Jika bukan, sistem akan memeriksa apakah pengguna memiliki \textit{work entry} “\textit{shift}” atau tidak. Jika iya, maka hanya fitur \textit{attendance system} yang akan ditampilkan, sedangkan jika tidak, akan terdapat alur tambahan untuk \textit{lunch system}.


\begin{figure}
    \centering
    \fbox{\includegraphics[width=1\linewidth]{assets/pics/dashboard fixed.png}}
    \caption{\textit{Flowchart Attendance System Fixed Type}}
    \label{fig:enter-label}
\end{figure}

\hspace{2em} Gambar 3.10 memperlihatkan \textit{flowchart} pada sistem \textit{attendance} dengan jenis \textit{work entry} “\textit{fixed}”. Proses dimulai dengan pengecekan apakah pengguna sudah melakukan \textit{clock-in}. Jika belum, sistem akan memeriksa keterlambatan. Apabila pengguna terlambat, mereka wajib mengisi alasan keterlambatan sebelum data disimpan ke dalam \textit{database}. Setelah \textit{clock-in}, sistem akan menampilkan fitur \textit{lunch system} dan memeriksa apakah pengguna sudah melakukan \textit{lunch in}. Jika belum, pengguna dapat melakukan \textit{lunch in}, kemudian \textit{lunch end}. Apabila \textit{lunch end} dilakukan terlambat, pengguna juga harus mengisi alasannya. Jika pengguna sudah melakukan \textit{clock-in} dan \textit{lunch in}, sistem akan memeriksa apakah pengguna mencoba \textit{clock-out} terlalu awal maka harus mengisi alasannya. Lalu, sistem akan menyimpan data ke dalam \textit{database}.

\begin{figure}
    \centering
    \fbox{\includegraphics[width=0.7\linewidth]{assets/pics/dashboard shift.png}}
    \caption{\textit{Flowchart Attendance System Shift Type}}
    \label{fig:enter-label}
\end{figure}

\hspace{2em} Gambar 3.11 memperlihatkan \textit{flowchart} pada sistem \textit{attendance} dengan jenis \textit{work entry} “\textit{shift}”. Proses dimulai dengan pengecekan apakah pengguna sudah melakukan \textit{clock-in}. Jika belum, sistem akan meminta jadwal \textit{shift} dan memeriksa keterlambatan. Apabila pengguna terlambat, mereka wajib mengisi alasan keterlambatan sebelum data disimpan ke dalam \textit{database}. Sebaliknya, jika pengguna sudah melakukan \textit{clock-in}, sistem akan memeriksa apakah pengguna mencoba \textit{clock-out} lebih awal. Jika iya, pengguna diwajibkan mengisi alasan sebelum data disimpan ke dalam \textit{database}.

\begin{figure}
    \centering
    \fbox{\includegraphics[width=0.9\linewidth]{assets/pics/event flow.png}}
    \caption{\textit{Flowchart Event Calendar}}
    \label{fig:enter-label}
\end{figure}

\hspace{2em} Gambar 3.12 menunjukkan \textit{flowchart} pada halaman \textit{Event Calendar}. Proses diawali dengan pemanggilan GET API untuk mengambil data \textit{event} dan hari libur yang telah tersimpan. Pengguna dapat melihat detail \textit{event} melalui modal yang menampilkan informasi lengkap beserta opsi untuk mengedit atau menghapus \textit{event}. Jika pengguna memilih untuk menghapus, sistem akan menampilkan modal konfirmasi terlebih dahulu, dan hanya pembuat \textit{event} yang memiliki hak untuk melakukan penghapusan. Selain itu, pengguna yang memiliki \textit{permission create} dapat membuat \textit{event} baru.

\begin{figure}
    \centering
    \fbox{\includegraphics[width=1\linewidth]{assets/pics/event create.png}}
    \caption{\textit{Flowchart Event Create}}
    \label{fig:enter-label}
\end{figure}

\hspace{2em} Gambar 3.13 memperlihatkan \textit{flowchart} pada halaman \textit{Create Event}. Proses dimulai dengan menampilkan \textit{form field} untuk mengisi data \textit{event}. Terdapat pilihan untuk \textit{event type}, seperti \textit{meeting} dan \textit{holiday}. Apabila pengguna memilih \textit{meeting}, maka data yang perlu diisi meliputi nama \textit{meeting}, tanggal, waktu pelaksanaan, daftar karyawan yang terlibat, warna yang akan ditampilkan pada kalender, deskripsi, dan dokumen pendukung. Sementara itu, jika memilih \textit{holiday,} pengguna hanya perlu mengisi tanggal, judul hari libur, serta \textit{optional field} berupa deskripsi dan dokumen pendukung. Setelah pengisian data selesai, akan ditampilkan modal konfirmasi. Apabila disetujui, data \textit{event} tersebut akan dikirimkan dan disimpan ke dalam \textit{database}.

\begin{figure}
    \centering
    \fbox{\includegraphics[width=1\linewidth]{assets/pics/event edit.png}}
    \caption{\textit{Flowchart Event Edit}}
    \label{fig:enter-label}
\end{figure}

\hspace{2em} Gambar 3.14 memperlihatkan \textit{flowchart} pada halaman \textit{Edit Event}. Proses dimulai dengan pemanggilan GET API untuk mengambil data \textit{event} yang dipilih untuk diubah. Pengguna dapat melakukan perubahan pada nama \textit{event}, waktu pelaksanaan, daftar karyawan yang terlibat, warna \textit{event}, deskripsi, dan dokumen pendukung. Setelah seluruh perubahan selesai dilakukan, akan ditampilkan modal konfirmasi. Apabila konfirmasi disetujui, data \textit{event} yang telah diperbarui akan dikirimkan dan disimpan ke dalam \textit{database}.
\\
\\


\section{\textit{Wireframe}}

\hspace{2em} Bagian ini menampilkan \textit{wireframe} sebagai rancangan dasar antarmuka \textit{Human Resource Internal System}. Rancangan \textit{wireframe} tersebut disajikan sebagai berikut:

\begin{figure}
    \centering
    \fbox{\includegraphics[width=0.8\linewidth]{assets/pics/wire contract list.png}}
    \caption{\textit{Wireframe Contract List}}
    \label{fig:enter-label}
\end{figure}

\hspace{2em} Gambar 3.15 adalah \textit{wireframe} pada halaman \textit{Contract List}. Pada halaman ini menampilkan tabel yang berisi seluruh daftar kontrak yang telah dibuat. Beberapa kolom yang ditampilkan meliputi \textit{Contract Name, Employee Name, Start Date, End Date, Job Title, Employee Status, Created At, Created By, Status,} dan \textit{Action}. Selain itu, tersedia tombol untuk membuat kontrak baru yang akan mengarahkan pengguna ke halaman \textit{Contract Input}. Pada bagian \textit{Action}, terdapat ikon detail (\textit{eye}) yang digunakan untuk memperbarui status atau melihat informasi lengkap kontrak, serta ikon \textit{edit} (\textit{pencil}) yang memungkinkan pengubahan data kontrak selama status kontrak masih \textit{pending}.

\begin{figure}
    \centering
    \fbox{\includegraphics[width=0.8\linewidth]{assets/pics/wire contract create.png}}
    \caption{\textit{Wireframe Contract Input}}
    \label{fig:enter-label}
\end{figure}

\hspace{2em} Gambar 3.16 adalah \textit{wireframe} pada halaman \textit{Contract Input}. Halaman ini digunakan untuk membuat kontrak baru bagi karyawan. Terdapat beberapa data yang harus dipenuhi yaitu pada bagian \textit{General Information} terdapat \textit{form fields employee name, contract name, job title, employee status, supervisor, work location, start date, end date,} dan \textit{leave}. Bagian \textit{Work Schedule} memuat \textit{form field} untuk \textit{work entry} dan \textit{work schedule}. Selanjutnya, pada bagian \textit{Attendance and Penalty Rule}, terdapat pengaturan \textit{Missing In Action} serta berbagai penalti lain yang dapat disesuaikan dengan kebijakan perusahaan. Terakhir, bagian \textit{Salary and Benefit} berisi informasi mengenai gaji pokok serta tambahan tunjangan atau manfaat lainnya.

\begin{figure}
    \centering
    \fbox{\includegraphics[width=0.8\linewidth]{assets/pics/wire contract detail.png}}
    \caption{\textit{Wireframe Contract Detail}}
    \label{fig:enter-label}
\end{figure}

\hspace{2em} Gambar 3.17 adalah \textit{wireframe} pada halaman \textit{Contract Detail}. Halaman ini menampilkan seluruh informasi kontrak yang telah dibuat dan menyediakan sejumlah aksi yang dapat dilakukan, seperti \textit{approve, reject, cancel,} atau \textit{terminate}, sesuai dengan status kontrak yang sedang berjalan.

\begin{figure}
    \centering
    \fbox{\includegraphics[width=0.8\linewidth]{assets/pics/wire payroll list.png}}
    \caption{\textit{Wireframe Payroll setting List}}
    \label{fig:enter-label}
\end{figure}

\hspace{2em} Gambar 3.18 adalah \textit{wireframe} pada halaman \textit{Payroll setting List}. Halaman ini menampilkan seluruh daftar \textit{item payroll} yang telah dibuat dan akan digunakan pada bagian\textit{ Salary and Benefit} saat proses pembuatan kontrak. Beberapa kolom yang ditampilkan meliputi \textit{Item Name, Description, Date Created,} dan \textit{Action}. Selain itu, tersedia tombol untuk menambahkan \textit{item payroll} baru yang akan mengarahkan pengguna ke halaman \textit{Payroll Setting Input}.

\begin{figure}
    \centering
    \fbox{\includegraphics[width=0.8\linewidth]{assets/pics/wire payroll create.png}}
    \caption{\textit{Wireframe Payroll setting Input}}
    \label{fig:enter-label}
\end{figure}

\hspace{2em} Gambar 3.19 adalah \textit{wireframe} pada halaman \textit{Payroll setting Input}. Halaman ini digunakan untuk menambahkan \textit{item payroll} baru, dan memungkinkan penambahan lebih dari satu data sekaligus melalui tombol \textit{Add Row}. Terdapat \textit{form fields} yang perlu diisi, yaitu \textit{item name} dan \textit{description}, sebelum item tersebut dapat disimpan ke dalam daftar \textit{payroll}.

\begin{figure}
    \centering
    \fbox{\includegraphics[width=0.8\linewidth]{assets/pics/wire payroll detail.png}}
    \caption{\textit{Wireframe Payroll setting Detail}}
    \label{fig:enter-label}
\end{figure}

\hspace{2em} Gambar 3.20 adalah \textit{wireframe} pada halaman \textit{Payroll setting Detail}.
Halaman ini menampilkan informasi lengkap mengenai \textit{item payroll} yang telah dibuat, seperti \textit{item name} dan \textit{description}.

\begin{figure}
    \centering
    \fbox{\includegraphics[width=0.8\linewidth]{assets/pics/wire attend fixed.png}}
    \caption{\textit{Wireframe Attendance work type Fixed}}
    \label{fig:enter-label}
\end{figure}

\hspace{2em} Gambar 3.21 adalah \textit{wireframe} pada halaman \textit{Dashboard attendance work type fixed}. Pada halaman ini, pengguna dapat melakukan proses absensi yang meliputi \textit{clock in, clock out}, serta absensi untuk jam makan siang sesuai dengan jadwal yang telah ditetapkan pada kontrak kerja dengan tipe \textit{fixed}.

\begin{figure}
    \centering
    \fbox{\includegraphics[width=0.8\linewidth]{assets/pics/wire attend shift.png}}
    \caption{\textit{Wireframe Attendance work type Shift}}
    \label{fig:enter-label}
\end{figure}

\hspace{2em} Gambar 3.22 adalah \textit{wireframe} pada halaman \textit{Dashboard attendance work type shift}. Pada halaman ini, pengguna dapat melakukan proses absensi dengan memilih jadwal \textit{shift} yang akan dilakukan terlebih dahulu, pilihan tersebut didapat berdasarkan jadwal yang telah ditetapkan pada kontrak kerja dengan tipe \textit{shift}.

\begin{figure}
    \centering
    \fbox{\includegraphics[width=0.8\linewidth]{assets/pics/wire event dash.png}}
    \caption{\textit{Wireframe Event Dashboard}}
    \label{fig:enter-label}
\end{figure}

\hspace{2em} Gambar 3.23 adalah \textit{wireframe} pada halaman \textit{Dashboard}. Pada halaman ini menampilkan daftar \textit{event} yang tersedia sebagai \textit{announcement} pada halaman utama, serta memungkinkan pengguna untuk melihat \textit{event} berdasarkan tanggal yang dipilih.

\begin{figure}
    \centering
    \fbox{\includegraphics[width=0.8\linewidth]{assets/pics/wire calendar view.png}}
    \caption{\textit{Wireframe Event Calendar}}
    \label{fig:enter-label}
\end{figure}

\hspace{2em} Gambar 3.24 adalah \textit{wireframe} pada halaman Event Calendar. Pada halaman ini menampilkan daftar jadwal dalam bentuk kalender dan memungkinkan pengguna untuk melihat tampilan jadwal berdasarkan mode \textit{month, week,} maupun \textit{day}, sehingga detail tanggal dan waktu kegiatan dapat dipantau dengan lebih mudah dan terstruktur.

\begin{figure}
    \centering
    \fbox{\includegraphics[width=0.8\linewidth]{assets/pics/wire calendar create.png}}
    \caption{\textit{Wireframe Event Input}}
    \label{fig:enter-label}
\end{figure}

\hspace{2em} Gambar 3.25 adalah \textit{wireframe} pada halaman \textit{Event Input}. Pada halaman ini digunakan untuk menambahkan \textit{event} baru dengan memilih jenis \textit{event} terlebih dahulu, seperti \textit{meeting, holiday,} atau \textit{schedule}. Setelah menentukan jenisnya, pengguna dapat mengisi sejumlah \textit{form field} yang tersedia, antara lain \textit{event name, date, start time, end time, color, guest, description,} serta mengunggah dokumen pendukung sebagai kelengkapan informasi terkait kegiatan yang akan dilaksanakan.

\begin{figure}
    \centering
    \fbox{\includegraphics[width=0.8\linewidth]{assets/pics/wire calendar create.png}}
    \caption{\textit{Wireframe Event Modal}}
    \label{fig:enter-label}
\end{figure}

\hspace{2em} Gambar 3.26 adalah \textit{wireframe} pada bagian \textit{Event Modal} pada halaman \textit{Event Calendar}. Bagian ini menampilkan detail lengkap dari jadwal yang dipilih, meliputi judul kegiatan, deskripsi, media atau \textit{file} terkait, daftar tamu yang diundang, serta tipe jadwal yang sedang ditampilkan.

\section{Implementasi}

\hspace{2em} \textit{Human Resource Internal System} telah berhasil dikembangkan dan diimplementasikan pada mode \textit{production}, serta resmi dijadikan produk yang digunakan oleh klien terkait. Berikut ini merupakan tampilan akhir dari \textit{Human Resource Internal System} yang telah selesai dikembangkan:

\begin{figure}
    \centering
    \fbox{\includegraphics[width=0.8\linewidth]{assets/pics/contract list view.png}}
    \caption{Tampilan \textit{Contract List}}
    \label{fig:enter-label}
\end{figure}

\hspace{2em} Gambar 3.27 menunjukkan tampilan halaman \textit{Contract List}. Pada halaman ini terdapat tabel yang menampilkan daftar kontrak yang telah dibuat. Beberapa kolom yang ditampilkan meliputi \textit{Contract Name, Employee Name, Start Date, End Date, Job Title, Employee Status, Created At, Created By, Status,} dan \textit{Action}. Jika \textit{user} ingin membuat kontrak baru, tersedia tombol yang mengarahkan ke halaman \textit{Contract Input}. Untuk kontrak yang berstatus pending, terdapat aksi \textit{Edit} (ikon \textit{pencil}) yang membawa \textit{user} ke halaman \textit{edit} kontrak untuk melakukan pembaruan atau penghapusan data. Selain itu, tersedia ikon detail (\textit{eye}) yang digunakan untuk memperbarui status kontrak atau melihat informasi lengkap dari kontrak tersebut.

\begin{figure}[H]
    \centering
    \begin{subfigure}[b]{0.8\linewidth}
        \centering
        \fbox{\includegraphics[width=0.8\linewidth]{assets/pics/contract input.png}}
    \end{subfigure}
    \hfill
    \begin{subfigure}[b]{0.8\linewidth}
        \centering
        \fbox{\includegraphics[width=0.8\linewidth]{assets/pics/contract input 2.png}}
    \end{subfigure}
    \caption{Tampilan \textit{Contract Input}}
    \label{fig:gambar-3.32}\textit{}
\end{figure}

\hspace{2em} Gambar 3.28 menunjukkan tampilan halaman \textit{Contract Input}. Pada halaman ini, \textit{user} dapat membuat kontrak baru. Tersedia beberapa \textit{form field} yang harus dipenuhi yaitu pada bagian \textit{General Information}, tersedia \textit{form field} seperti \textit{employee name, contract name, job title, employee status, supervisor, work location, start date, end date,} dan \textit{leave}. Bagian\textit{ Work Schedule} mencakup \textit{form field} untuk \textit{work entry} dan \textit{work schedule}. Selanjutnya, pada bagian \textit{Attendance and Penalty Rule}, terdapat pengaturan mengenai \textit{Missing In Action} yang wajib di isi serta berbagai jenis penalti lain yang dapat disesuaikan dengan kebijakan perusahaan. Terakhir, bagian \textit{Salary and Benefit} memuat informasi terkait gaji pokok yang diharuskan di isi dengan \textit{unit addition} serta tambahan tunjangan atau manfaat lainnya.

\begin{figure}
    \centering
    \fbox{\includegraphics[width=0.8\linewidth]{assets/pics/contract detail view.png}}
    \caption{Tampilan \textit{Contract Detail}}
    \label{fig:enter-label}
\end{figure}

\hspace{2em} Gambar 3.29 menunjukkan tampilan halaman \textit{Contract Detail}. Pada halaman ini menampilkan informasi lengkap dari kontrak yang telah dibuat dan menyediakan beberapa opsi tindakan sesuai dengan status kontrak. Jika kontrak berstatus \textit{pending}, tersedia pilihan \textit{approve} dan \textit{reject}. Apabila kontrak berstatus \textit{approved}, pengguna dapat melakukan \textit{cancel contract}. Sementara itu, untuk kontrak dengan status \textit{on-going}, tersedia opsi \textit{terminate}. Jika kontrak berstatus \textit{rejected}, maka kontrak tersebut dianggap tidak berlaku dan harus dibuat ulang melalui pembuatan kontrak baru.

\begin{figure}
    \centering
    \fbox{\includegraphics[width=0.8\linewidth]{assets/pics/payroll setting list.png}}
    \caption{Tampilan \textit{Payroll setting List}}
    \label{fig:enter-label}
\end{figure}

\begin{figure}
    \centering
    \fbox{\includegraphics[width=0.8\linewidth]{assets/pics/payroll setting delete.png}}
    \caption{Tampilan \textit{Payroll setting List} Modal Delete}
    \label{fig:enter-label}
\end{figure}

\hspace{2em} Gambar 3.30 menunjukkan tampilan halaman \textit{Payroll Setting List}. Halaman ini menampilkan daftar \textit{item payroll} yang telah dibuat, dengan beberapa kolom seperti \textit{Item Name, Description, Date Created,} dan \textit{Action}. Tersedia tombol \textit{Add Payroll Item} untuk menambahkan \textit{item payroll} baru. Selain itu, pada setiap item terdapat aksi \textit{Edit} (ikon \textit{pencil}) yang akan mengarahkan pengguna ke halaman \textit{Payroll Edit}, serta aksi \textit{Delete} (ikon tempat sampah) yang menampilkan sebuah modal konfirmasi seperti terlihat pada Gambar 3.31. Namun, untuk item Gaji Pokok/\textit{Main Salary}, seluruh aksi tersebut tidak dapat digunakan karena item tersebut bersifat wajib (\textit{required}) dan tidak dapat diubah maupun dihapus.

\begin{figure}[H]
    \centering
    \begin{subfigure}[b]{0.8\linewidth}
        \centering
        \fbox{\includegraphics[width=0.8\linewidth]{assets/pics/payroll setting input.png}}
    \end{subfigure}
    \hfill
    \begin{subfigure}[b]{0.8\linewidth}
        \centering
        \fbox{\includegraphics[width=0.8\linewidth]{assets/pics/payroll setting input 2.png}}
    \end{subfigure}
    \caption{Tampilan \textit{Payroll Setting Input}}
    \label{fig:gambar-3.32}\textit{}
\end{figure}

\hspace{2em} Gambar 3.32 menunjukkan tampilan halaman \textit{Payroll Setting Input}. Pada halaman ini, \textit{user} dapat membuat item \textit{payroll} baru yang nantinya dapat dipilih saat pembuatan kontrak pada bagian \textit{salary and benefit}. Tersedia beberapa \textit{form field}, yaitu \textit{Item Name} dan \textit{Description}. Selain itu, terdapat tombol \textit{Add Row} jika \textit{user} ingin menambahkan lebih dari satu item dalam satu kali input seperti pada tampilan Gambar 3.33.

\begin{figure}
    \centering
    \fbox{\includegraphics[width=0.8\linewidth]{assets/pics/attendance fixed.png}}
    \caption{Tampilan \textit{Attendance work type Fixed}}
    \label{fig:enter-label}
\end{figure}

\hspace{2em} Gambar 3.34 menunjukkan tampilan halaman \textit{Dashboard}. Pada halaman ini, \textit{user} dapat melakukan sistem \textit{attendance} dengan \textit{work type Fixed}, sehingga akan muncul \textit{card} tambahan \textit{Lunch Attendance} apabila terdapat jadwal \textit{lunch} pada kontrak yang telah dibuat. Sistem \textit{attendance} akan aktif berdasarkan jadwal yang ditetapkan di dalam kontrak, dan fitur \textit{Lunch Time} akan aktif setelah \textit{user} melakukan \textit{Clock In}.

\begin{figure}
    \centering
    \fbox{\includegraphics[width=0.8\linewidth]{assets/pics/attendance shift.png}}
    \caption{Tampilan \textit{Attendance work type Shift}}
    \label{fig:enter-label}
\end{figure}

\hspace{2em} Gambar 3.35 menunjukkan tampilan halaman \textit{Dashboard}. Pada halaman ini, \textit{user} dapat melakukan \textit{attendance} dengan \textit{work type Shift}, sehingga tersedia \textit{dropdown} untuk memilih jadwal \textit{shift} sebelum melakukan \textit{Clock In}. Sistem akan menolak proses \textit{Clock In} apabila \textit{user} tidak memilih \textit{shift} yang sesuai dengan waktu masuk yang ditentukan.

\begin{figure}[H]
    \centering
    \begin{subfigure}[b]{0.8\linewidth}
        \centering
        \fbox{\includegraphics[width=\linewidth]{assets/pics/dashboard before.png}}
        \caption*{(a) Tampilan \textit{Before}}
    \end{subfigure}
    \hfill
    \begin{subfigure}[b]{0.8\linewidth}
        \centering
        \fbox{\includegraphics[width=\linewidth]{assets/pics/dashboard after.png}}
        \caption*{(b) Tampilan \textit{After}}
    \end{subfigure}
    \caption{Tampilan halaman \textit{Dashboard} \textit{Before} dan \textit{After}}
    \label{fig:gambar-3.32}\textit{}
\end{figure}

\hspace{2em} Gambar 3.36 (a) menunjukkan tampilan \textit{Dashboard} sebelum ditambahkan fitur \textit{Event Announcement}. Pada tampilan sebelumnya, belum tersedia informasi terkait acara yang berlangsung, sehingga menyulitkan \textit{user} untuk mengetahui acara apa saja yang ada pada hari tertentu dan mengharuskan mereka membuka halaman \textit{Event} terlebih dahulu.

\hspace{2em} Gambar 3.36 (b) menunjukkan tampilan \textit{Dashboard} setelah penambahan fitur \textit{Event Announcement}. Fitur ini berfungsi untuk menampilkan informasi acara yang diadakan berdasarkan hari yang dipilih, sehingga memudahkan \textit{user} untuk mengetahui acara yang tersedia maupun yang baru ditambahkan tanpa perlu membuka halaman \textit{Event} terlebih dahulu. Selain itu, fitur ini telah dilengkapi dengan implementasi Websocket untuk memberikan pembaruan informasi acara secara \textit{real-time}, sehingga setiap perubahan dapat langsung terlihat oleh pengguna.

\begin{figure}[H]
    \centering
    \begin{subfigure}[b]{0.8\linewidth}
        \centering
        \fbox{\includegraphics[width=\linewidth]{assets/pics/event calendar.png}}
        \caption*{(a) Tampilan Jadwal \textit{Event}}
    \end{subfigure}
    \hfill
    \begin{subfigure}[b]{0.8\linewidth}
        \centering
        \fbox{\includegraphics[width=\linewidth]{assets/pics/event calendar collapse.png}}
        \caption*{(b) Tampilan Jadwal \textit{Event collapse}}
    \end{subfigure}
    \caption{Tampilan halaman \textit{Event Calendar}}
    \label{fig:gambar-3.32}\textit{}
\end{figure}

\hspace{2em} Gambar 3.37 (a) menunjukkan tampilan \textit{Event Calendar}. Pada halaman ini menampilkan seluruh kegiatan yang telah dibuat sesuai dengan tanggal dan waktu yang telah ditetapkan. Jika terdapat lebih dari dua kegiatan pada tanggal yang sama, tersedia tombol \textit{See More} yang akan mengarahkan pengguna ke tampilan mode \textit{Day} untuk melihat seluruh kegiatan pada hari tersebut. Selain itu, apabila terdapat jadwal yang saling bertumpukan atau \textit{collapse} pada waktu yang sama, sistem akan menampilkan sebuah modal berisi daftar kegiatan yang waktunya saling bertabrakan, seperti ditunjukkan pada Gambar 3.37 (b).

\hspace{2em} Pada menu \textit{Event} di \textit{sidebar}, terdapat \textit{badge notification} yang menunjukkan jumlah total \textit{event} yang belum dilihat atau dibuka detailnya oleh pengguna. Di dalam kalender, \textit{event} yang belum dibuka juga ditandai dengan sebuah titik (\textit{dot}) berwarna putih sebagai indikator bahwa detail kegiatan tersebut masih belum dilihat.

\begin{figure}[H]
    \centering
    \begin{subfigure}[b]{0.8\linewidth}
        \centering
        \fbox{\includegraphics[width=\linewidth]{assets/pics/event modal meeting.png}}
        \caption*{(a) Tampilan modal detail \textit{Event type Meeting} atau \textit{Schedule}}
    \end{subfigure}
    \hfill
    \begin{subfigure}[b]{0.8\linewidth}
        \centering
        \fbox{\includegraphics[width=\linewidth]{assets/pics/event modal holiday.png}}
        \caption*{(b) Tampilan modal detail \textit{Event type Holiday}}
    \end{subfigure}
    \caption{Tampilan halaman \textit{Event Calendar} modal detail}
    \label{fig:gambar-3.32}\textit{}
\end{figure}

\hspace{2em} Gambar 3.38 (a) menunjukkan tampilan modal detail \textit{Event Calendar} untuk tipe \textit{meeting} atau \textit{schedule}. Pada bagian ini ditampilkan informasi lengkap mengenai kegiatan, seperti judul \textit{event}, deskripsi, media atau dokumen terkait, tanggal dan waktu pelaksanaan, daftar tamu yang diundang, serta tipe \textit{event}.

\hspace{2em} Sementara itu, Gambar 3.38 (b) menampilkan modal detail \textit{Event Calendar} untuk tipe \textit{holiday}. Modal ini hanya berisi informasi judul hari libur, deskripsi, serta tanggal pelaksanaan. Untuk tipe \textit{holiday}, \textit{event} secara otomatis dikirimkan ke seluruh pengguna dan diatur berlangsung sepanjang hari (\textit{all day}).

\hspace{2em} Selain itu, terdapat tombol aksi \textit{Edit} (ikon \textit{pencil}) yang mengarahkan pengguna ke halaman \textit{}Event Edit untuk memperbarui kegiatan, serta aksi \textit{Delete} (ikon tempat sampah) untuk menghapus \textit{event}. Aksi ini hanya dapat dilakukan oleh pengguna yang membuat kegiatan tersebut, sedangkan pengguna yang diundang hanya dapat melihat detail informasi tanpa dapat melakukan perubahan.

\begin{figure}[H]
    \centering
    \begin{subfigure}[b]{0.8\linewidth}
        \centering
        \fbox{\includegraphics[width=\linewidth]{assets/pics/event input before.png}}
        \caption*{(a) Tampilan \textit{Before}}
    \end{subfigure}
    \hfill
    \begin{subfigure}[b]{0.8\linewidth}
        \centering
        \fbox{\includegraphics[width=\linewidth]{assets/pics/event input after.png}}
        \caption*{(b) Tampilan \textit{After}}
    \end{subfigure}
    \caption{Tampilan halaman \textit{Event Input} \textit{Before} dan \textit{After}}
    \label{fig:gambar-3.32}\textit{}
\end{figure}

\hspace{2em} Gambar 3.39 (a) menunjukkan tampilan halaman \textit{Event Input} sebelum dilakukan \textit{revamp}. Pada halaman tersebut, \textit{user} dapat membuat \textit{event}, namun belum tersedia fitur untuk menambahkan \textit{supporting document} atau \textit{description} yang diperlukan untuk memberikan detail tambahan mengenai \textit{event} yang akan dibuat.

\hspace{2em} Gambar 3.39 (b) menunjukkan tampilan halaman \textit{Event Input} setelah dilakukan \textit{revamp} pada bagian \textit{form input}. Pada halaman ini ditambahkan \textit{form field Description} untuk memberikan catatan terkait \textit{event}, serta fitur \textit{Upload Media} untuk mengunggah \textit{file} yang diperlukan, seperti \textit{file} PPT untuk keperluan \textit{meeting}.

\begin{figure}
    \centering
    \fbox{\includegraphics[width=0.8\linewidth]{assets/pics/event input holiday.png}}
    \caption{Tampilan halaman \textit{Event Input type Holiday}}
    \label{fig:enter-label}
\end{figure}

\hspace{2em} Gambar 3.40 menunjukkan tampilan halaman \textit{Event Input} dengan tipe holiday. Pada bagian \textit{form input} untuk tipe ini, hanya terdapat \textit{form field Event Name} untuk memberikan nama kegiatan libur, tanggal pelaksanaan yang ditetapkan, serta \textit{form field opsional} seperti \textit{description} dan upload document sebagai kelengkapan informasi terkait hari libur tersebut.

\section{Kendala dan Solusi yang Ditemukan}

\hspace{2em} Beberapa kendala yang ditemui selama melaksanakan kegiatan magang di PT Visi Karya Nusantara adalah sebagai berikut:

\begin{enumerate}
    \item Terjadinya ketidaksesuaian komunikasi antara tim \textit{backend} dan \textit{frontend} pada tahap awal magang, yang menyebabkan terjadinya miskomunikasi dan memerlukan penyesuaian tambahan saat proses integrasi.
    
    \item Kesulitan dalam menentukan alur fitur atau modul yang akan dikembangkan, sehingga membutuhkan waktu lebih untuk melakukan analisis serta riset yang mendalam.
\end{enumerate}

\hspace{2em} Adapun solusi yang diterapkan untuk mengatasi kendala tersebut adalah sebagai berikut:

\begin{enumerate}
    \item Meningkatkan kualitas komunikasi antar tim \textit{backend} dan \textit{frontend} guna memastikan kesesuaian saat proses integrasi.
    
    \item  Memperdalam pemahaman terhadap konsep fitur yang dikembangkan serta melakukan riset lebih lanjut terkait kebutuhan dan ekspektasi pengguna terhadap fitur atau modul tersebut.
\end{enumerate}