%-----------------------------------------------------------------------------%
\chapter{\babSatu}
%-----------------------------------------------------------------------------%

%-----------------------------------------------------------------------------%
\section{Latar Belakang Masalah}
%-----------------------------------------------------------------------------%
\hspace{2em} Seiring kemajuan teknologi, perannya semakin penting dalam berbagai aspek kehidupan, termasuk dunia kerja dan aktivitas sehari-hari. Dalam bidang profesional, teknologi membantu mempermudah penyelesaian tugas sekaligus meningkatkan efisiensi kerja \cite{Abbas2022}. Salah satu bentuk teknologi yang berkembang pesat adalah \textit{website}, yang berfungsi sebagai sarana informasi dan memudahkan penyebaran berbagai hal kepada masyarakat luas \cite{Arafat2022}. Sektor perkantoran menjadi salah satu yang terdampak digitalisasi. Dengan adanya perkembangan teknologi informasi, penyelesaian pekerjaan kantor pun menuntut pendekatan yang lebih modern \cite{Rohiyatun2020}. Dalam konteks perkantoran, pemanfaatan teknologi berbasis \textit{website} mendorong digitalisasi berbagai proses administratif, termasuk dalam pengelolaan sumber daya manusia. Penerapan \textit{Human Resource Information System} (HRIS) memungkinkan perusahaan mengelola data karyawan dan memantau kinerja secara efisien dan terstruktur melalui platform berbasis \textit{website} \cite{Nugraha2022}.

\hspace{2em} HRIS sendiri merupakan sistem yang mengintegrasikan data serta proses yang berkaitan dengan sumber daya manusia ke dalam satu platform berbasis komputer \cite{Thakur2023}. Melalui sistem ini, perusahaan dapat mengatur informasi karyawan, riwayat pekerjaan, pengembangan karir, hingga kebutuhan administrasi lainnya secara terpadu \cite{Hijrasil2023}. Dari sisi efektivitas, HRIS juga berkontribusi dalam meningkatkan kinerja karyawan, yang pada akhirnya berdampak positif terhadap keberhasilan organisasi. Implementasinya terbukti memiliki peran penting dalam mendukung produktivitas dan efektivitas kerja di berbagai industri \cite{Manunggal2022}.

\hspace{2em} Penerapan fitur \textit{Contract} dalam HRIS mendukung manajemen dalam mengambil keputusan yang lebih tepat serta mempercepat komunikasi antara karyawan dan HR. Transformasi layanan HR dari berbasis kertas menjadi \textit{self-service} juga meningkatkan efisiensi departemen \cite{Jahan2014HRIS}. Informasi penting seperti tanggal berakhirnya kontrak dapat diakses dengan mudah sehingga perencanaan tenaga kerja di setiap unit menjadi lebih terarah \cite{Tursunbayeva2020Planned}. 

\pagebreak
\hspace{2em} Kehadiran fitur ini dirancang untuk memastikan data hubungan kerja terdokumentasi dengan jelas, mengurangi risiko kesalahan administrasi, serta menjamin kepatuhan terhadap regulasi ketenagakerjaan. Dengan demikian, \textit{contract} menjadi komponen krusial dalam HRIS yang mendorong transparansi, akuntabilitas, dan efektivitas pengelolaan SDM.

\hspace{2em} Fitur \textit{Payroll Setting} ditambahkan pada sistem untuk memudahkan admin dalam mengatur komponen penggajian yang bersifat fleksibel, seperti tunjangan makan, transportasi, maupun benefit lainnya. Data yang diinput melalui fitur ini nantinya akan terintegrasi dengan halaman \textit{Contract} untuk menentukan detail gaji dan fasilitas karyawan secara lebih jelas dan terstruktur.

\hspace{2em} Adanya dikembangkan sebuah \textit{Attendance System} yang mencakup absensi \textit{clock in, clock out,} dan \textit{lunch}, yang disesuaikan dengan data kontrak karyawan yang telah tercatat. Dengan demikian, sistem absensi dapat digunakan baik untuk karyawan dengan jadwal \textit{shift} maupun \textit{fixed hours}. Waktu dan hari absensi juga akan otomatis mengikuti ketentuan yang tercantum dalam kontrak masing-masing karyawan.

\hspace{2em} Fitur \textit{Event} dikembangkan untuk mempermudah pengguna dalam memantau dan mengelola jadwal kegiatan yang akan datang. Pada fitur ini, pengguna dapat menambahkan informasi lengkap mengenai kegiatan, seperti deskripsi event dan dokumen pendukung yang relevan. Setiap \textit{event} akan ditampilkan secara otomatis pada \textit{dashboard} masing-masing pengguna yang diundang sesuai dengan kegiatan tersebut, serta diperbarui secara \textit{real-time}. Dengan demikian, fitur ini membantu memastikan setiap pengguna tetap terinformasi dan dapat mempersiapkan dengan baik terhadap jadwal atau aktivitas yang telah dijadwalkan.

% PT Visi Karya Nusantara adalah perusahaan yang bergerak di bidang \textit{software house} sejak Agustus 2025. Saat ini, perusahaan memiliki sekitar 10 karyawan dari berbagai divisi. Salah satu fokus pengembangan utama adalah pembuatan HRIS (\textit{Human Resource Information System}) sebagai produk untuk mendukung proses pengelolaan SDM, khususnya dalam mengatur sistem kontrak dan payroll agar lebih efisien.

\hspace{2em} Oleh karena itu, kegiatan magang ini diarahkan pada pengembangan HRIS dengan menekankan pada fitur \textit{contract}, \textit{payroll system}, \textit{attendance system}, dan \textit{event}. Dalam proses pembuatan \textit{website}, digunakan \textit{framework} yang dapat mempercepat sekaligus mempermudah pengembangan. Pada sisi \textit{frontend}, sistem dibangun menggunakan \textit{framework} ReactJS dengan TypeScript.

% %-----------------------------------------------------------------------------%
% \section{Permasalahan}
% %-----------------------------------------------------------------------------%
% Pada bagian ini akan dijelaskan mengenai definisi permasalahan 
% yang \saya~hadapi dan ingin diselesaikan serta asumsi dan batasan 
% yang digunakan dalam menyelesaikannya.


%-----------------------------------------------------------------------------%
\section{Maksud dan Tujuan Kerja Magang}

\hspace{2em} Adapun maksud pelaksanaan magang di PT Visi Karya Nusantara adalah untuk menambahkan dan mengembangkan modul \textit{contract, payroll system,} \textit{attendance system}, dan \textit{event} ke dalam \textit{Human Resource Information System }(HRIS) perusahaan. Tujuan dari pengembangan ini meliputi sebagai berikut:
\begin{enumerate}
    \item Meningkatkan\textit{ hard skill }dan \textit{soft skill} sebagai \textit{Front end Developer} berdasarkan pengalaman yang diperoleh selama perkuliahan.
    \item Memperluas wawasan serta mendapatkan pengalaman langsung dalam lingkungan kerja profesional.
    \item Meningkatkan kemampuan komunikasi, kerja sama tim, adaptasi, dan manajemen waktu dalam situasi kerja nyata.
    \item Membangun jaringan profesional yang dapat mendukung pengembangan karir di masa depan.
\end{enumerate}
%-----------------------------------------------------------------------------%

%-----------------------------------------------------------------------------%
\section{Waktu dan Prosedur Pelaksanaan Kerja Magang}
%-----------------------------------------------------------------------------%
\hspace{2em} Program magang dilaksanakan selama lima bulan, mulai dari 1 September 2025 hingga 1 Februari 2026, sesuai dengan kontrak kerja yang telah disepakati dengan perusahaan PT Visi Karya Nusantara yang berlokasi di Start Space Coworking Space Gading Serpong. Pelaksanaan magang ini didampingi oleh seorang pembimbing lapangan atau \textit{supervisor}, yaitu  Atanasius Raditya Herkristito yang berperan sebagai \textit{Senior Software Engineer} di PT Visi Karya Nusantara. Waktu program magang di PT Visi Karya Nusantara adalah sebagai berikut:

\begin{enumerate}
    \item Jadwal dan Metode Magang
    Magang dilaksanankan setiap Senin hingga Jumat mulai pukul 09.00 WIB
    hingga 18.00 WIB dengan sistem WFH (\textit{Work From Home}). Komunikasi antar karyawan dan tim dilakukan melalui aplikasi Discord sebagai media utama koordinasi.
    
    \item Rapat dan Evaluasi
    Setiap hari Kamis diadakan rapat rutin yang mencakup \textit{progress review, code review,} serta pembahasan kendala yang dihadapi selama pengerjaan proyek. Selain itu, dilakukan evaluasi antar tim untuk memberikan masukan dan meningkatkan kualitas kerja.
    
\end{enumerate}



