%---------------------------------------------------------------
\chapter{\babEmpat}
%---------------------------------------------------------------

%---------------------------------------------------------------
\section{Simpulan}
%---------------------------------------------------------------
\hspace{2em} Pengembangan \textit{website Human Resource Information System} (HRIS) pada PT Visi Karya Nusantara bertujuan untuk mempermudah tim \textit{Human Resource} dalam melakukan koordinasi, pengelolaan data, serta monitoring aktivitas karyawan secara menyeluruh. Seluruh proses yang sebelumnya dilakukan secara manual kini telah terintegrasi dalam satu sistem, sehingga alur kerja menjadi lebih efisien, terstruktur, dan mudah diakses kapan pun dibutuhkan. \textit{Website} ini telah sepenuhnya dikembangkan dan dipublikasikan dalam mode \textit{production}, sehingga dapat langsung digunakan oleh klien untuk mendukung berbagai kebutuhan operasional seperti koordinasi internal, pemantauan kehadiran, pengajuan dokumen, hingga pengelolaan kontrak dan jadwal kegiatan.

\hspace{2em} Dalam proses pengembangannya, HRIS dibangun menggunakan framework React TypeScript pada sisi \textit{frontend} guna memastikan tampilan antarmuka yang responsif, interaktif, dan mudah digunakan. Sementara itu, sisi \textit{backend} dikembangkan menggunakan ExpressJS untuk menyediakan layanan API yang stabil, cepat, dan terstruktur. Hasil akhir dari pengembangan HRIS mencakup beberapa modul utama, yaitu \textit{Contract, Payroll System, Event,} serta \textit{Dashboard} yang di dalamnya terdapat fitur \textit{Attendance}. Setiap modul dirancang untuk saling terintegrasi sehingga mampu mendukung keseluruhan proses bisnis perusahaan secara efektif dan berkelanjutan.
%---------------------------------------------------------------
\section{Saran}
%---------------------------------------------------------------

\hspace{2em} Selama pelaksanaan kegiatan magang di PT Visi Karya Nusantara, terdapat beberapa saran yang dapat diberikan untuk mendukung pengembangan sistem kedepannya. Adapun saran-saran tersebut adalah sebagai berikut:

\begin{enumerate}
    \item Penambahan fitur kasbon untuk mempermudah tim \textit{Human Resource} dalam melakukan pendataan karyawan yang mengajukan pinjaman kepada perusahaan, sekaligus menjadi bukti resmi terkait pengajuan pinjaman tersebut.
    \pagebreak

    \item Melakukan integrasi jadwal \textit{holiday event} pada fitur \textit{attendance} dan pengajuan cuti, sehingga karyawan tidak tercatat sebagai \textit{missing in action} apabila tidak melakukan absensi pada hari yang telah ditetapkan sebagai hari libur. Selain itu, pengajuan cuti juga tidak akan berkurang apabila tanggal tersebut termasuk dalam hari libur.

    \item Menambahkan sistem penggajian pada modul \textit{Contract} yang mendukung perhitungan harian, mingguan, bulanan, maupun tahunan, serta dilengkapi dengan perhitungan gaji pokok secara otomatis berdasarkan interval yang dipilih.
\end{enumerate}