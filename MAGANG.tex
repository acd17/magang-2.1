% @author  unknown
% @version 1.1a
% @edit by Andreas Febrian
% @edit by Moeljono Widjaja
% @edit by Arya Wicaksana
%
% Template Laporan Skripsi Informatika UMN
%

%
% Tipe dokumen adalah report dengan satu kolom. 
%
\documentclass[12pt, a4paper, onecolumn, oneside, final]{report1}

% \usepackage{indentfirst}

% Load konfigurasi LaTeX untuk tipe laporan thesis
\usepackage{_internals/umnthesis}
\usepackage{pdfpages}
\usepackage{hyperref}
\usepackage{ragged2e}
\usepackage{enumitem}
\usepackage{array}



% Daftar pemenggalan suku kata dan istilah dalam LaTeX
\input{_internals/hype.indonesia}

% Load konfigurasi khusus untuk laporan yang sedang dibuat
%-----------------------------------------------------------------------------%
% Informasi Mengenai Dokumen
%-----------------------------------------------------------------------------%
% 
% Judul laporan. 
\var{\judul}{Pengembangan Front-end Human Resource Information System pada PT Visi Karya Nusantara}
% Judul Pendek (Tiga Kata Pertama). 
\var{\judulPendek}{Pengembangan Front-end}
% 
% Tulis kembali judul laporan, kali ini akan diubah menjadi huruf kapital
% \Var{\Judul}{Judul Tugas Akhir}

\Var{\Judul}{Pengembangan Front-end Human Resource Information System pada PT Visi Karya Nusantara}
% 
% Tulis kembali judul laporan namun dengan bahasa Ingris 
\var{\judulInggris}{Front-End Development of the Human Resource Information System at PT Visi Karya Nusantara}



% 
% Tipe laporan:
% Laporan MBKM Magang
% Laporan MBKM Kewirausahaan
% Laporan MBKM Penelitian
% Laporan MBKM Proyek Desa
% Laporan MBKM Independent

\var{\type}{LAPORAN CAREER ACCELERATION PROGRAM}
% 
% Tulis kembali tipe laporan, kali ini akan diubah menjadi huruf kapital
\Var{\Type}{LAPORAN CAREER ACCELERATION PROGRAM}
% 
% Tulis nama penulis 
\var{\penulis}{Amanda Citra Dewanti}
% 
% Tulis kembali nama penulis, kali ini akan diubah menjadi huruf kapital
\Var{\Penulis}{Amanda Citra Dewanti}

% Informasi tempat kerja magang
\var{\namaPerusahaan}{PT Visi Karya Nusantara}
\var{\divisi}{Software Engineer}
\var{\alamatPerusahaan}{Start Space Coworking Space Gading Serpong, Tangerang, Banten, 15810}
\var{\periodeMagang}{1 September 2025 - 1 February 2026}
\var{\pembimbingLapangan}{Atanasius Raditya Herkristito}

% 
% Tulis NIM penulis
\var{\nim}{00000066344}
%
% Tuliskan Jenjang Studi penulis (D3/S1/S2)
\Var{\Jenjang}{S1}
\var{\jenjang}{S1}
% 
% Tuliskan Fakultas dimana penulis berada
\Var{\Fakultas}{Teknik dan Informatika}
\var{\fakultas}{Teknik dan Informatika}
% 
% Tuliskan Program Studi yang diambil penulis
\Var{\Program}{INFORMATIKA}
\var{\program}{Informatika}
% 
% Tuliskan tahun publikasi laporan
\Var{\bulan}{December}
\Var{\tahun}{2025}
% 
% Tuliskan gelar yang akan diperoleh dengan menyerahkan laporan ini
\var{\gelar}{Sarjana Ilmu Komputer}
% 
% Tuliskan tanggal pengesahan laporan
\var{\tanggalPengesahan}{15 December 2025} 


% Tuliskan tanggal pengumpulan laporan, waktu dimana laporan diserahkan ke 
% penguji/sekretariat
\var{\tanggalPengumpulan}{15 December 2025} 


% Tuliskan tanggal sidang
\var{\hariTanggalSidang}{Senin?, Tgl. Sidang?} 
\var{\waktuSidang}{00.00 s/s 00.00} 

% 
% Tuliskan tanggal keputusan sidang dikeluarkan dan penulis dinyatakan 
% lulus/tidak lulus
\var{\tanggalLulus}{Tgl. Lulus}
% 


% Tuliskan Rektor UMN
\var{\rektorUMN}{ Dr. Ir. Andrey Andoko, M.Sc.}

% Tuliskan Dekan FTI
\var{\dekanFTI}{Dr.   Eng.   Niki  Prastomo,  S.T.,  M.Sc.}

% Tuliskan Kaprodi Informatika 
\var{\kaprodi}{Arya Wicaksana, S.Kom., M.Eng.Sc., OCA}
\var{\kaprodiNIDN}{ NIDN: 0315109103}

% Tuliskan pembimbing tunggal atau ke-1
\var{\pembimbing}{Eunike Endariahna Surbakti, S.Kom., M.T.I.}
% Tuliskan NIDN pembimbing tunggal atau ke-1
\var{\pembimbingNIDN}{NIDN: 0012345600}

% Tuliskan pembimbing ke-2
\var{\pembimbingb}{Nama Lengkap Beserta Gelar}
% Tuliskan NIDN pembimbing ke-2
\var{\pembimbingbNIDN}{NIDN: 0012345600}

% Tuliskan ketua sidang
\var{\ketuaSidang}{Nama Lengkap Beserta Gelar}

% Tuliskan dosen penguji
\var{\penguji}{Nama Lengkap Beserta Gelar}
\var{\pengujiNIDN}{NIDN: 0123456789}
% 
% Alias untuk memudahkan alur penulisan pada saat menulis laporan
\var{\saya}{Penulis}

%-----------------------------------------------------------------------------%
% Judul Setiap Bab
%-----------------------------------------------------------------------------%
% 
% Berikut ada judul-judul setiap bab. 
% Silahkan diubah sesuai dengan kebutuhan. 
% 
\Var{\kataPengantar}{Kata Pengantar}
\Var{\babSatu}{Pendahuluan}
\Var{\babDua}{Gambaran Umum Perusahaan}
\Var{\babTiga}{Pelaksanaan Kerja Magang}
\Var{\babEmpat}{Simpulan dan Saran}

% Daftar istilah yang mungkin perlu ditandai 
\input{istilah}



% Awal bagian penulisan laporan
\begin{document}
\hyphenpenalty=10000
%
% Sampul Laporan
\input{src/1.Awal/00-SampulJudul}

%
% Gunakan penomeran romawi
\pagenumbering{roman}
% setelah bagian ini, halaman dihitung sebagai halaman ke 2
\setcounter{page}{1}

\BgThispage



\pagestyle{fancy}

% Halaman Judul
\addChapter{HALAMAN JUDUL}
\onehalfspacing
    \begin{center}    
    
        % \vspace*{1.0cm}
        % judul thesis harus dalam 14pt Times New Roman

        \begin{minipage}{0.9\textwidth}
            \centering
            \bo{\large \Judul} 
        \end{minipage}
        

        
        \vspace*{0.5cm}
        \begin{figure}
            \begin{center}
                % \includegraphics[width=2.5cm]{_internals/makara.eps}
                \includegraphics[width=5cm]{assets/pics/logo_UMN_clean.png}
            \end{center}
        \end{figure}    
        % \vspace*{0.5cm}        % harus dalam 14pt Times New Roman
        \MakeUppercase{ \large \Type} 
        \vspace*{5cm}
               
        
        % Diajukan sebagai salah satu syarat untuk memperoleh\\
        % Gelar Sarjana Komputer (S.Kom.) \\[1cm]
        % penulis dan npm
        \MakeUppercase{\large \bo{\penulis}} \\
        \bo{\large \nim} \\

        \vfill

        % informasi mengenai fakultas dan program studi
        \bo{\large
        	PROGRAM STUDI \Program \\
        	FAKULTAS \Fakultas\\
        	UNIVERSITAS MULTIMEDIA NUSANTARA\\
        	TANGERANG \\
        	\tahun
        }
    \end{center}
\newpage


\addChapter{HALAMAN PERNYATAAN ORISINALITAS}
% Nonaktifkan watermark dan footer sementara
% \vspace{-2cm}
\chapter*{HALAMAN PERNYATAAN ORISINALITAS TIDAK PLAGIAT}

\noindent
Dengan ini saya,

\noindent
\begin{tabular}{lcl}
   Nama  &:& \penulis \\
   NIM  &:& \nim \\
   Program Studi &:& \program \\
%   Fakultas &:& \fakultas
\end{tabular}

\vspace{\baselineskip}

\noindent
Menyatakan dengan sesungguhnya bahwa \type \ saya yang berjudul:

\noindent
\bo{\judul}

\vspace{\baselineskip}
\noindent
merupakan hasil karya saya sendiri, bukan merupakan hasil plagiat, dan tidak pula dituliskan oleh orang lain; Semua sumber, baik yang dikutip maupun dirujuk, telah saya cantumkan dan nyatakan dengan benar pada bagian Daftar Pustaka.

\vspace{\baselineskip}
\noindent
Jika di kemudian hari terbukti ditemukan kecurangan/penyimpangan, baik dalam pelaksanaan skripsi maupun dalam penulisan laporan karya ilmiah, saya bersedia menerima konsekuensi untuk dinyatakan TIDAK LULUS. Saya juga bersedia menanggung segala konsekuensi hukum yang berkaitan dengan tindak plagiarisme ini sebagai kesalahan saya pribadi dan bukan tanggung jawab Universitas Multimedia Nusantara.


\vspace{1cm}


\begin{flushright}
Tangerang, \tanggalPengumpulan \\[1.5cm]

\includegraphics[width=0.2\textwidth]{assets/pics/ttd acd.png} \\[1cm]

(\penulis)
\end{flushright}


\newpage

\clearpage

\addChapter{HALAMAN PERNYATAAN PENGGUNAAN BANTUAN KECERDASAN ARTIFISIAL (AI)}
% \vspace{-2cm}
\chapter*{HALAMAN PERNYATAAN PENGGUNAAN BANTUAN KECERDASAN ARTIFISIAL (AI)}

\noindent
Saya yang bertanda tangan di bawah ini:

\noindent
\begin{tabular}{lcl}
   Nama  &:& \penulis \\
   NIM  &:& \nim \\
   Program Studi &:& \program \\
   Judul Laporan  &:& \begin{minipage}[t]{0.75\textwidth} \noindent \justifying \judul \end{minipage} \\
%   Fakultas &:& \fakultas
\end{tabular}

\vspace{\baselineskip}

\noindent
Dengan ini saya menyatakan secara jujur menggunakan bantuan Kecerdasan Artifisial (AI) dalam pengerjaan Laporan sebagai berikut :

\noindent
\begin{itemize}[itemsep=0pt, topsep=0pt, parsep=0pt, partopsep=0pt]
    \renewcommand{\labelitemi}{${\rlap{\hspace{0.1em}$\checkmark$}}\square$}
    % \renewcommand{\labelitemi}{$\square$}
    \item Menggunakan AI sebagaimana diizinkan untuk membantu dalam menghasilkan ide-ide utama saja

    % \renewcommand{\labelitemi}{${\rlap{\hspace{0.1em}$\checkmark$}}\square$}
    \renewcommand{\labelitemi}{$\square$}
    \item Menggunakan AI sebagaimana diizinkan untuk membantu menghasilkan teks pertama saja 

    \renewcommand{\labelitemi}{$\square$}
    \item Menggunakan AI untuk menyempurnakan sintaksis dan tata bahasa untuk pengumpulan tugas

    \renewcommand{\labelitemi}{$\square$}
    \item Karena tidak diizinkan: Tidak menggunakan bantuan AI dengan cara apa pun dalam pembuatan tugas
\end{itemize}

% \vspace{\baselineskip}

\noindent
Saya juga menyatakan bahwa:
\noindent
\begin{enumerate}[label=(\arabic*), itemsep=0pt, topsep=0pt, parsep=0pt]
    \item Menyerahkan secara lengkap dan jujur penggunaan perangkat AI yang diperlukan dalam tugas melalui Formulir Penggunaan Perangkat Kecerdasan Artifisial (AI) 
    \item Mengakui telah menggunakan bantuan AI dalam tugas saya baik dalam bentuk kata, paraphrase, penyertaan ide atau fakta penting yang disarankan oleh AI dan saya telah menyantumkan dalam sitasi serta referensi
    \item Terlepas dari pernyataan di atas, tugas ini sepenuhnya merupakan karya saya sendiri
\end{enumerate}
\begin{flushright}
Tangerang, \tanggalPengumpulan \\

\includegraphics[width=0.2\textwidth]{assets/pics/ttd acd.png} \\

(\penulis)
\end{flushright}

\newpage
\clearpage

\addChapter{HALAMAN PERNYATAAN KEABSAHAN PERUSAHAAN}
% \vspace{-2cm}
\chapter*{HALAMAN PERNYATAAN KEABSAHAN PERUSAHAAN}

\noindent
\begin{tabular}{lcl}
   Nama  &:& \penulis \\
   NIM  &:& \nim \\
   Program Studi &:& \program \\
   Fakultas &:& \fakultas \\
\end{tabular}

\vspace{\baselineskip}

\noindent
menyatakan bahwa saya melaksanakan kegiatan di:

\vspace{\baselineskip}

\noindent
\vspace{\baselineskip}
\begin{tabular}{lcl}
   Nama Perusahaan/Organisasi  &:& PT Visi Karya Nusantara \\
   Alamat  &:&  \begin{minipage}[t]{0.55\textwidth} \noindent \justifying Start Space Coworking Space Gading Serpong, Tangerang, Banten, 15810 \end{minipage} \\
   Email Perusahaan/Organisasi &:&  aftersixidn@gmail.com \\
\end{tabular}

\noindent
\begin{enumerate}
    \item Perusahaan/Organisasi tempat saya melakukan kegiatan dapat di validasi keberadaannya.
    \item Jika dikemudian hari, terbukti ditemukan kecurangan/penyimpangan data yang tidak valid di perusahaan/organisasi tempat saya melakukan kegiatan, maka:
    \begin{enumerate}[label=\alph*.]
        \item Saya bersedia menerima konsekuensi dinyatakan TIDAK LULUS untuk mata kuliah yang telah saya tempuh.
        \item Saya bersedia menerima semua sanksi yang berlaku sebagaimana ditetapkan dalam peraturan yang berlaku di Universitas Multimedia Nusantara.

    \end{enumerate}

\end{enumerate}

\noindent
Pernyataan ini saya buat dengan sebenar-benarnya dan digunakan sebagaimana mestinya


% \vspace{0.25cm}


\begin{flushright}
Tangerang, 19 November 2025
\end{flushright}

\vspace*{0.5cm}

% \noindent
% \begin{tabular}{p{0.5\textwidth} p{0.5\textwidth}}
% \centering Mengetahui & \centering Menyatakan \\
% \vspace{4cm} & \vspace{4cm} \\  % ruang untuk tanda tangan
% \centering (Andi Wahyudi) & \centering (Muhammad Affransyah Bayulaksana) \\
% \end{tabular}

\noindent
\begin{minipage}{.5\textwidth}
\begin{center}
    % Pembimbing 
    \begin{flushleft}
    Mengetahui \\

        \includegraphics[width=0.35\textwidth]{assets/pics/ttd spv.jpg} \\
        

    Atanasius Raditya Herkristito \\ 
        
    \end{flushleft}
    % \pembimbingNIDN
\end{center}
  
  
  
\end{minipage}% This must go next to `\end{minipage}`
\begin{minipage}{.5\textwidth}
\begin{center}
    \begin{flushright}
    Menyatakan \\

\includegraphics[width=0.3\textwidth]{assets/pics/ttd acd.png} \\

    (\penulis)\\  
    \end{flushright}
      
  % \pengujiNIDN
    
\end{center}
\end{minipage}


\newpage
\clearpage



%
% load halaman persetujuan (jika untuk maju sidang)
% \addChapter{HALAMAN PERSETUJUAN}
% % \clearpage
% \chapter*{HALAMAN PENGESAHAN}
\onehalfspacing
% \vspace*{0.4cm}

\begin{center}
   \type \, dengan judul \\[0.5cm]
    
    \bo{\Judul}  \\[1cm]
    oleh \\[0.5cm]



\noindent
\begin{tabular}{l l p{6cm}}
	Nama&: & \penulis \\
	NIM&: & \nim \\
	Program Studi&: & Informatika \\
	Fakultas &: & Fakultas Teknik dan Informatika \\
\end{tabular} \\

\vspace{1em}


Telah diujikan pada hari \hariTanggalSidang \\
Pukul \waktuSidang \ dan dinyatakan \\
LULUS \\
Dengan susunan penguji sebagai berikut \\


\end{center}

\vspace*{1.0cm}

\noindent
\begin{minipage}{.5\textwidth}
\begin{center}
    % Pembimbing 
    Dosen Pembimbing \\[2.5cm]
    
    
    (\pembimbing) \\
    \pembimbingNIDN
\end{center}
  
  
  
\end{minipage}% This must go next to `\end{minipage}`
\begin{minipage}{.5\textwidth}
\begin{center}
  Penguji \\[2.5cm]
  
  (\penguji)\\
  \pengujiNIDN
    
\end{center}
\end{minipage}

\vspace*{0.2cm}
\begin{center}
     
%     % Pembimbing Tunggal
%     Pembimbing \\[2cm]
    
    
%     (\pembimbing)
    
    
% %     % Pembimbing I dan II
% % \begin{minipage}{.5\textwidth}
% %     \begin{center}
% %         Pembimbing I\\[2cm]
    
    
% %     (\pembimbing)
        
% %     \end{center}
% % \end{minipage}% This must go next to `\end{minipage}`
% % \begin{minipage}{.5\textwidth}
% %     \begin{center}
% %             Pembimbing II\\[2cm]
    
    
% %     (\pembimbingb) \\[0.5cm]
        
% %     \end{center}
% % \end{minipage}


\vspace{2em}
    Ketua Program Studi \program, \\[2.5cm]
    
    
    (\kaprodi)\\
    \kaprodiNIDN
    
    
\end{center}


\newpage
% % \thispagestyle{empty}
% % \begingroup
% % \renewcommand{\BgThispage}{} % Matikan watermark
% % \includepdf[
% %     pages=1,
% %     pagecommand={},
% %     width=\paperwidth,
% %     % pagecommand={\thispagestyle{empty}} % Nonaktifkan footer
% % ]{assets/pdf/LembarPersetujuan_00000077007.pdf}
% % \endgroup
% \clearpage

% %
% % load halaman pengesahan (jika sesudah selesai sidang) menggantikan halaman persetujuan
% \addChapter{HALAMAN PENGESAHAN}
% \chapter*{HALAMAN PENGESAHAN}
\onehalfspacing
% \vspace*{0.4cm}

\begin{center}
   \type \, dengan judul \\[0.5cm]
    
    \bo{\Judul}  \\[1cm]
    oleh \\[0.5cm]



\noindent
\begin{tabular}{l l p{6cm}}
	Nama&: & \penulis \\
	NIM&: & \nim \\
	Program Studi&: & Informatika \\
	Fakultas &: & Fakultas Teknik dan Informatika \\
\end{tabular} \\

\vspace{1em}


Telah diujikan pada hari \hariTanggalSidang \\
Pukul \waktuSidang \ dan dinyatakan \\
LULUS \\
Dengan susunan penguji sebagai berikut \\


\end{center}

\vspace*{1.0cm}

\noindent
\begin{minipage}{.5\textwidth}
\begin{center}
    % Pembimbing 
    Dosen Pembimbing \\[2.5cm]
    
    
    (\pembimbing) \\
    \pembimbingNIDN
\end{center}
  
  
  
\end{minipage}% This must go next to `\end{minipage}`
\begin{minipage}{.5\textwidth}
\begin{center}
  Penguji \\[2.5cm]
  
  (\penguji)\\
  \pengujiNIDN
    
\end{center}
\end{minipage}

\vspace*{0.2cm}
\begin{center}
     
%     % Pembimbing Tunggal
%     Pembimbing \\[2cm]
    
    
%     (\pembimbing)
    
    
% %     % Pembimbing I dan II
% % \begin{minipage}{.5\textwidth}
% %     \begin{center}
% %         Pembimbing I\\[2cm]
    
    
% %     (\pembimbing)
        
% %     \end{center}
% % \end{minipage}% This must go next to `\end{minipage}`
% % \begin{minipage}{.5\textwidth}
% %     \begin{center}
% %             Pembimbing II\\[2cm]
    
    
% %     (\pembimbingb) \\[0.5cm]
        
% %     \end{center}
% % \end{minipage}


\vspace{2em}
    Ketua Program Studi \program, \\[2.5cm]
    
    
    (\kaprodi)\\
    \kaprodiNIDN
    
    
\end{center}


\newpage



\addChapter{HALAMAN PERSETUJUAN PUBLIKASI KARYA ILMIAH}
\chapter*{HALAMAN PERSETUJUAN PUBLIKASI KARYA ILMIAH MAHASISWA}

\onehalfspacing

% \vspace*{0.2cm}
\noindent 
Yang bertanda tangan di bawah ini:
% \vspace*{0.4cm}

\begin{tabular}{p{3.7cm} l p{6.5cm}}
	Nama & : & \penulis \\ 	
	NIM & : & \nim \\
	Program Studi & : & \program\\	
	Jenjang & : & \jenjang\\
	Jenis Karya & : & \type \\
\end{tabular}

% \vspace*{0.6cm}
\noindent Menyatakan dengan sesungguhnya bahwa:
\begin{itemize}
    \renewcommand{\labelitemi}{${\rlap{\hspace{0.1em}$\checkmark$}}\square$}
    % \renewcommand{\labelitemi}{$\square$}
    \item Saya bersedia memberikan izin sepenuhnya kepada Universitas Multimedia Nusantara untuk mempublikasikan hasil karya ilmiah saya di repositori Knowledge Center, sehingga dapat diakses oleh Civitas Akademika/Publik. Saya menyatakan bahwa karya ilmiah yang saya buat tidak mengandung data yang bersifat konfidensial dan saya juga tidak akan mencabut kembali izin yang telah saya berikan dengan alasan apapun.

    % \renewcommand{\labelitemi}{${\rlap{\hspace{0.1em}$\checkmark$}}\square$}
    \renewcommand{\labelitemi}{$\square$}
    \item Saya tidak bersedia karena dalam proses pengajuan untuk diterbitkan ke jurnal/konferensi nasional/internasional (dibuktikan dengan \textit{letter of acceptance})**.
\end{itemize}

\begin{flushright}
Tangerang, \tanggalPengumpulan \\

\includegraphics[width=0.3\textwidth]{assets/pics/ttd acd.png}

\noindent
\penulis
\end{flushright}

\onehalfspacing

\vfill
{
\footnotesize
\noindent ** Jika tidak bisa membuktikan LoA jurnal/HKI selama enam bulan ke depan, saya bersedia mengizinkan penuh karya ilmiah saya untuk diunggah ke KC UMN dan menjadi hak institusi UMN.
}
\newpage
\clearpage


\addChapter{HALAMAN PERSEMBAHAN/MOTO}
\input{src/1.Awal/05-Persembahan}
\clearpage


\addChapter{KATA PENGANTAR}
\chapter*{KATA PENGANTAR}
% \onehalfspacing

Puji syukur saya panjatkan ke hadirat Tuhan Yang Maha Esa atas terselesaikannya penulisan Laporan Magang berjudul "Pengembangan \textit{Front-End Human Resource Information System} pada PT Visi Karya Nusantara.” Laporan ini disusun sebagai salah satu syarat untuk memperoleh gelar Sarjana Komputer pada Program Studi Informatika, Fakultas Teknik dan Informatika, Universitas Multimedia Nusantara. Saya menyadari bahwa proses penyusunan laporan ini tidaklah mudah. Dalam pelaksanaannya, saya memperoleh banyak bantuan, dukungan, dan arahan dari berbagai pihak. Untuk itu, saya menyampaikan ucapan terima kasih kepada:

\noindent Mengucapkan terima kasih
\begin{enumerate}
	\item Bapak \rektorUMN, selaku Rektor Universitas Multimedia
Nusantara. 
	\item Bapak \dekanFTI, selaku Dekan Fakultas Teknik dan Informatika Universitas Multimedia Nusantara.
	\item Bapak \kaprodi, selaku Ketua Program Studi Informatika Universitas Multimedia Nusantara. 
	\item Ibu \pembimbing,  sebagai Pembimbing magang yang telah banyak meluangkan
    waktu untuk memberikan bimbingan, arahan dan motivasi atas terselesainya laporan magang ini.
    \item Kepada Pimpinan Perusahaan, Pembimbing Lapangan, dan Anggota Tim yang telah membimbing dan membantu sehingga laporan ini dapat terselesaikan
\end{enumerate}
Semoga laporan magang ini dapat memberikan manfaat, baik sebagai referensi informasi maupun sebagai sumber inspirasi bagi pembaca.

\vspace*{0.1cm}

\begin{flushright}
Tangerang, \tanggalPengumpulan \\[0.1cm]

\includegraphics[width=0.15\textwidth]{assets/pics/ttd acd.png}

\penulis
\end{flushright}
\clearpage


\addChapter{ABSTRAK}
%-----------------------------------------------------------------------------%
\chapter*{\Judul}
%-----------------------------------------------------------------------------%
\singlespacing
\begin{center}
    
    \vspace{-4em}
    
    \penulis
    
	\bigskip
    
    \textbf{ABSTRAK}
    
\end{center}

% \chapter*{Abstrak}

\vspace*{0.2cm}
{
	\setlength{\parindent}{0pt}

	\bigskip
	\bigskip

Perkembangan teknologi memberikan dampak yang signifikan di berbagai bidang, termasuk dalam dunia perkantoran. Salah satu penerapannya pada bidang sumber daya manusia adalah \textit{Human Resource Information System} (HRIS), yang dirancang untuk meningkatkan efisiensi dan efektivitas dalam pengelolaan karyawan. PT Visi Karya Nusantara mengembangkan sistem HRIS untuk mendigitalisasi proses-proses penting, terutama pengelolaan kontrak, komponen gaji, hingga penjadwalan acara. Sistem ini dibangun menggunakan ReactJS TypeScript pada sisi \textit{frontend} serta ExpressJS pada sisi \textit{backend}. Fitur yang dikembangkan mencakup modul \textit{Contract, Payroll System, Attendance}, dan \textit{Event}. HRIS yang telah dibangun kini telah menjadi produk dan berada pada tahap \textit{production}, sehingga dapat digunakan oleh \textit{client} dalam operasional sehari-hari. Ke depannya, sistem ini diharapkan terus berkembang untuk mendukung peningkatan produk serta memenuhi kebutuhan \textit{client} secara berkelanjutan.
	\bigskip
 
% Kata kunci urut abjad
% 3 – 5 kata kunci
	\textbf{Kata kunci:} {\textit{Human Resource Information System, Production, ReactJS TypeScript} } 

\onehalfspacing
\clearpage

\addChapter{ABSTRACT}
%-----------------------------------------------------------------------------%
\chapter*{\MakeUppercase{\textit{\judulInggris}}}
%-----------------------------------------------------------------------------%
\singlespacing
\begin{center}
    
    \vspace{-4em}
    
    \penulis
    
	\bigskip
    
    \textit{\textbf{ABSTRACT}}
    
\end{center}

% \chapter*{Abstrak}

\vspace*{0.2cm}
{
	\setlength{\parindent}{0pt}

	\bigskip
	\bigskip


\textit{The rapid growth of technology has created a significant impact across various sectors, including the corporate environment. One of its applications in the human resource field is the Human Resource Information System (HRIS), which is designed to improve efficiency and effectiveness in managing employees. PT Visi Karya Nusantara developed an HRIS to digitalize key processes, particularly contract management, payroll components, and event scheduling. The system is built using ReactJS TypeScript on the frontend and ExpressJS on the backend. The features developed include the Contract, Payroll System, Attendance, and Event modules. The HRIS has now become a product and is currently in the production stage, allowing clients to use it in their daily operational activities. In the future, the system is expected to continue evolving to support product improvements and sustainably meet client needs.}
	\bigskip
 
% keywords in alphabetical order
% 3 – 5 keywords
	\textbf{Keywords:} {\textit{Human Resource Information System, Production, ReactJS TypeScript} } 

\onehalfspacing
\clearpage


%
% Daftar isi, gambar, tabel, dan kode
%
\singlespacing
\phantomsection
%  \pagestyle{daftarIsi}
\pagestyle{plain}
\tableofcontents
\clearpage



\phantomsection
\listoftables
\clearpage

\phantomsection
\listoffigures
\clearpage

% \phantomsection
% \lstlistoflistings
% \clearpage


% \phantomsection
% \listofequation
% \addChapter{DAFTAR RUMUS}
% \clearpage


% \phantomsection
% \listoflistings
% \addChapter{DAFTAR KODE}
% \clearpage





\phantomsection
\listofappendices
\addChapter{DAFTAR LAMPIRAN}
\clearpage

\onehalfspacing

\pagestyle{fancy}

%
% Gunakan penomeran Arab (1, 2, 3, ...) setelah bagian ini.
%
\pagenumbering{arabic}

%
% Isi Skripsi
%

\BgThispage

%-----------------------------------------------------------------------------%
\chapter{\babSatu}
%-----------------------------------------------------------------------------%

%-----------------------------------------------------------------------------%
\section{Latar Belakang Masalah}
%-----------------------------------------------------------------------------%
\hspace{2em} Seiring kemajuan teknologi, perannya semakin penting dalam berbagai aspek kehidupan, termasuk dunia kerja dan aktivitas sehari-hari. Dalam bidang profesional, teknologi membantu mempermudah penyelesaian tugas sekaligus meningkatkan efisiensi kerja \cite{Abbas2022}. Salah satu bentuk teknologi yang berkembang pesat adalah \textit{website}, yang berfungsi sebagai sarana informasi dan memudahkan penyebaran berbagai hal kepada masyarakat luas \cite{Arafat2022}. Sektor perkantoran menjadi salah satu yang terdampak digitalisasi. Dengan adanya perkembangan teknologi informasi, penyelesaian pekerjaan kantor pun menuntut pendekatan yang lebih modern \cite{Rohiyatun2020}. Dalam konteks perkantoran, pemanfaatan teknologi berbasis \textit{website} mendorong digitalisasi berbagai proses administratif, termasuk dalam pengelolaan sumber daya manusia. Penerapan \textit{Human Resource Information System} (HRIS) memungkinkan perusahaan mengelola data karyawan dan memantau kinerja secara efisien dan terstruktur melalui platform berbasis \textit{website} \cite{Nugraha2022}.

\hspace{2em} HRIS sendiri merupakan sistem yang mengintegrasikan data serta proses yang berkaitan dengan sumber daya manusia ke dalam satu platform berbasis komputer \cite{Thakur2023}. Melalui sistem ini, perusahaan dapat mengatur informasi karyawan, riwayat pekerjaan, pengembangan karir, hingga kebutuhan administrasi lainnya secara terpadu \cite{Hijrasil2023}. Dari sisi efektivitas, HRIS juga berkontribusi dalam meningkatkan kinerja karyawan, yang pada akhirnya berdampak positif terhadap keberhasilan organisasi. Implementasinya terbukti memiliki peran penting dalam mendukung produktivitas dan efektivitas kerja di berbagai industri \cite{Manunggal2022}.

\hspace{2em} Penerapan fitur \textit{Contract} dalam HRIS mendukung manajemen dalam mengambil keputusan yang lebih tepat serta mempercepat komunikasi antara karyawan dan HR. Transformasi layanan HR dari berbasis kertas menjadi \textit{self-service} juga meningkatkan efisiensi departemen \cite{Jahan2014HRIS}. Informasi penting seperti tanggal berakhirnya kontrak dapat diakses dengan mudah sehingga perencanaan tenaga kerja di setiap unit menjadi lebih terarah \cite{Tursunbayeva2020Planned}. 

\pagebreak
\hspace{2em} Kehadiran fitur ini dirancang untuk memastikan data hubungan kerja terdokumentasi dengan jelas, mengurangi risiko kesalahan administrasi, serta menjamin kepatuhan terhadap regulasi ketenagakerjaan. Dengan demikian, \textit{contract} menjadi komponen krusial dalam HRIS yang mendorong transparansi, akuntabilitas, dan efektivitas pengelolaan SDM.

\hspace{2em} Fitur \textit{Payroll Setting} ditambahkan pada sistem untuk memudahkan admin dalam mengatur komponen penggajian yang bersifat fleksibel, seperti tunjangan makan, transportasi, maupun benefit lainnya. Data yang diinput melalui fitur ini nantinya akan terintegrasi dengan halaman \textit{Contract} untuk menentukan detail gaji dan fasilitas karyawan secara lebih jelas dan terstruktur.

\hspace{2em} Adanya dikembangkan sebuah \textit{Attendance System} yang mencakup absensi \textit{clock in, clock out,} dan \textit{lunch}, yang disesuaikan dengan data kontrak karyawan yang telah tercatat. Dengan demikian, sistem absensi dapat digunakan baik untuk karyawan dengan jadwal \textit{shift} maupun \textit{fixed hours}. Waktu dan hari absensi juga akan otomatis mengikuti ketentuan yang tercantum dalam kontrak masing-masing karyawan.

\hspace{2em} Fitur \textit{Event} dikembangkan untuk mempermudah pengguna dalam memantau dan mengelola jadwal kegiatan yang akan datang. Pada fitur ini, pengguna dapat menambahkan informasi lengkap mengenai kegiatan, seperti deskripsi event dan dokumen pendukung yang relevan. Setiap \textit{event} akan ditampilkan secara otomatis pada \textit{dashboard} masing-masing pengguna yang diundang sesuai dengan kegiatan tersebut, serta diperbarui secara \textit{real-time}. Dengan demikian, fitur ini membantu memastikan setiap pengguna tetap terinformasi dan dapat mempersiapkan dengan baik terhadap jadwal atau aktivitas yang telah dijadwalkan.

% PT Visi Karya Nusantara adalah perusahaan yang bergerak di bidang \textit{software house} sejak Agustus 2025. Saat ini, perusahaan memiliki sekitar 10 karyawan dari berbagai divisi. Salah satu fokus pengembangan utama adalah pembuatan HRIS (\textit{Human Resource Information System}) sebagai produk untuk mendukung proses pengelolaan SDM, khususnya dalam mengatur sistem kontrak dan payroll agar lebih efisien.

\hspace{2em} Oleh karena itu, kegiatan magang ini diarahkan pada pengembangan HRIS dengan menekankan pada fitur \textit{contract}, \textit{payroll system}, \textit{attendance system}, dan \textit{event}. Dalam proses pembuatan \textit{website}, digunakan \textit{framework} yang dapat mempercepat sekaligus mempermudah pengembangan. Pada sisi \textit{frontend}, sistem dibangun menggunakan \textit{framework} ReactJS dengan TypeScript.

% %-----------------------------------------------------------------------------%
% \section{Permasalahan}
% %-----------------------------------------------------------------------------%
% Pada bagian ini akan dijelaskan mengenai definisi permasalahan 
% yang \saya~hadapi dan ingin diselesaikan serta asumsi dan batasan 
% yang digunakan dalam menyelesaikannya.


%-----------------------------------------------------------------------------%
\section{Maksud dan Tujuan Kerja Magang}

\hspace{2em} Adapun maksud pelaksanaan magang di PT Visi Karya Nusantara adalah untuk menambahkan dan mengembangkan modul \textit{contract, payroll system,} \textit{attendance system}, dan \textit{event} ke dalam \textit{Human Resource Information System }(HRIS) perusahaan. Tujuan dari pengembangan ini meliputi sebagai berikut:
\begin{enumerate}
    \item Meningkatkan\textit{ hard skill }dan \textit{soft skill} sebagai \textit{Front end Developer} berdasarkan pengalaman yang diperoleh selama perkuliahan.
    \item Memperluas wawasan serta mendapatkan pengalaman langsung dalam lingkungan kerja profesional.
    \item Meningkatkan kemampuan komunikasi, kerja sama tim, adaptasi, dan manajemen waktu dalam situasi kerja nyata.
    \item Membangun jaringan profesional yang dapat mendukung pengembangan karir di masa depan.
\end{enumerate}
%-----------------------------------------------------------------------------%

%-----------------------------------------------------------------------------%
\section{Waktu dan Prosedur Pelaksanaan Kerja Magang}
%-----------------------------------------------------------------------------%
\hspace{2em} Program magang dilaksanakan selama lima bulan, mulai dari 1 September 2025 hingga 1 Februari 2026, sesuai dengan kontrak kerja yang telah disepakati dengan perusahaan PT Visi Karya Nusantara yang berlokasi di Start Space Coworking Space Gading Serpong. Pelaksanaan magang ini didampingi oleh seorang pembimbing lapangan atau \textit{supervisor}, yaitu  Atanasius Raditya Herkristito yang berperan sebagai \textit{Senior Software Engineer} di PT Visi Karya Nusantara. Waktu program magang di PT Visi Karya Nusantara adalah sebagai berikut:

\begin{enumerate}
    \item Jadwal dan Metode Magang
    Magang dilaksanankan setiap Senin hingga Jumat mulai pukul 09.00 WIB
    hingga 18.00 WIB dengan sistem WFH (\textit{Work From Home}). Komunikasi antar karyawan dan tim dilakukan melalui aplikasi Discord sebagai media utama koordinasi.
    
    \item Rapat dan Evaluasi
    Setiap hari Kamis diadakan rapat rutin yang mencakup \textit{progress review, code review,} serta pembahasan kendala yang dihadapi selama pengerjaan proyek. Selain itu, dilakukan evaluasi antar tim untuk memberikan masukan dan meningkatkan kualitas kerja.
    
\end{enumerate}




%-----------------------------------------------------------------------------%
\chapter{\babDua}
%-----------------------------------------------------------------------------%

%-----------------------------------------------------------------------------%
\section{Sejarah Singkat Perusahaan}
%-----------------------------------------------------------------------------%
\hspace{2em} PT Visi Karya Nusantara, yang lebih dikenal dengan AfterSix, didirikan pada tahun 2025 dan bergerak di bidang \textit{software house}. Perusahaan ini berlokasi di Start Space Coworking Space Gading Serpong. Saat ini, perusahaan tengah mengerjakan berbagai proyek di antaranya Nine to Six, AKS, IkoVisual, Beeliv, dan KAIYA. Salah satu identitas perusahaan dapat dilihat pada Gambar 2.1, yang menampilkan logo resmi PT Visi Karya Nusantara.

\begin{figure}
    \centering
    \fbox{\includegraphics[width=0.4\linewidth]{assets/pics/Frame 5490.png}}
    \caption{Logo perusahaan PT Visi Karya Nusantara}
      {\small Sumber: \cite{PTVisiKaryaNusantara}}
    \label{fig:logo}
\end{figure}


%-----------------------------------------------------------------------------%
\section{Visi dan Misi Perusahaan}
%-----------------------------------------------------------------------------%
\hspace{2em} Terdapat visi dan misi dari PT Visi Karya Nusantara, yaitu \cite{PTVisiKaryaNusantara}: 
\subsection{Visi}
\hspace{2em} Menjadi perusahaan konsultan teknologi informasi yang memberikan solusi yang optimal, efektif, dan efisien, serta membangun budaya perusahaan yang berintegritas, profesional, dan berkelanjutan.

\subsection{Misi}

\begin{enumerate}
    \item Menyediakan layanan konsultasi teknologi informasi yang berorientasi pada efektivitas dan efisiensi, guna memastikan setiap solusi yang diberikan selaras dengan kebutuhan dan tujuan klien.

    \item Membangun organisasi yang profesional dan berintegritas, melalui pengembangan sumber daya manusia yang adaptif, kompeten, dan berdaya saing, serta menumbuhkan budaya kerja yang kolaboratif, inovatif, dan berkelanjutan.

    \item Berperan aktif dalam mendukung transformasi digital di Indonesia, dengan mengedepankan praktik konsultasi yang beretika, berkualitas, dan berorientasi pada hasil jangka panjang.
\end{enumerate}

%-----------------------------------------------------------------------------%
\section{Struktur Organisasi Perusahaan}
%-----------------------------------------------------------------------------%

\hspace{2em} Struktur organisasi di PT Visi Karya Nusantara dapat dilihat pada Gambar 2.2, yang menggambarkan hubungan antar posisi serta alur koordinasi dalam perusahaan \cite{PTVisiKaryaNusantara}.

\begin{figure}
    \centering
     \fbox{\includegraphics[width=1 \linewidth]{assets/pics/stucture.jpg}}
   \caption{Struktur organisasi pada PT Visi Karya Nusantara}
   {\small Sumber: \cite{PTVisiKaryaNusantara}}
  \label{fig:enter-label}
\end{figure}

\pagebreak
\hspace{2em} Struktur organisasi tersebut terdiri dari beberapa peran utama yang memiliki tanggung jawab masing-masing dalam mendukung operasional perusahaan. Pada tingkat tertinggi, \textit{Director} bertanggung jawab atas pengambilan keputusan strategis serta memastikan seluruh kegiatan perusahaan berjalan sesuai visi dan misi yang telah ditetapkan. Di bawahnya terdapat \textit{Software Engineer Manager} yang berperan dalam mengelola tim teknis, memantau produktivitas, serta mengoordinasikan alur kerja antara \textit{Software Engineer} dan \textit{UI/UX Designer} agar setiap proyek dapat diselesaikan secara efektif dan tepat waktu.

\hspace{2em} Selanjutnya, \textit{Software Engineer} memiliki tanggung jawab dalam merancang, mengembangkan, dan memelihara sistem perangkat lunak sehingga fungsionalitas sistem tetap optimal dan sesuai kebutuhan perusahaan. Pada sisi lain, \textit{UI/UX Designer} bertugas merancang tampilan dan pengalaman pengguna yang intuitif, menarik, dan mudah digunakan untuk memastikan kenyamanan pengguna dalam berinteraksi dengan sistem. Setiap peran dalam struktur organisasi ini saling berkoordinasi dan bekerja sama untuk mencapai tujuan perusahaan secara optimal serta mendukung pengembangan produk yang berkualitas.




\chapter{\babTiga}

%-----------------------------------------------------------------------------%

%-----------------------------------------------------------------------------%
\section{Kedudukan dan Koordinasi}
%-----------------------------------------------------------------------------%
\hspace{2em} Selama pelaksanaan kegiatan magang di PT Visi Karya Nusantara, posisi yang ditempati adalah Software Engineer di bawah pengawasan dan bimbingan Atanasius Raditya Herkristito. Tugas utama selama magang adalah mengembangkan \textit{Human Resource Information System} (HRIS), yang mencakup perancangan dan implementasi antarmuka untuk meningkatkan efisiensi sistem dalam pengelolaan data kepegawaian.

\begin{figure}
    \centering
     \fbox{\includegraphics[width=0.6 \linewidth]{assets/pics/stucture internal.jpg}}
   \caption{Struktur organisasi Pengembang}
\end{figure}

\hspace{2em} Dalam pengembangan HRIS, tim terdiri dari satu \textit{Software Engineer Manager}, satu \textit{UI/UX Designer}, dan empat \textit{Software Engineer}. Selama pengembangan, komunikasi dan koordinasi dilakukan melalui aplikasi Discord. Tinjauan progres dan \textit{code review} dilakukan setiap minggu, tepatnya di hari Kamis. Selain itu, terdapat sesi evaluasi dengan seluruh tim untuk memberikan \textit{feedback} terkait perkembangan proyek, guna meningkatkan kualitas berikutnya.

\section{Tugas yang Dilakukan}

\hspace{2em} Selama Pelaksanaan magang di PT Visi Karya Nusantara, tugas utama yang dikerjakan adalah mengembangkan \textit{Human Resource Information System} (HRIS), yang direncanakan menjadi produk resmi dan digunakan oleh perusahaan. Adapun tugas yang dilakukan meliputi:

\begin{enumerate}
    \item Membangun \textit{frontend} pada modul \textit{contract} dan \textit{payroll system}
    \item Mengembangkan sistem absensi yang mencakup fitur \textit{clock in, clock out}, dan \textit{lunch} pada modul \textit{attendance}.
    \item Mengembangkan modul \textit{Event} yang mencakup proses pembuatan kegiatan serta pengumuman (announce) secara langsung pada \textit{dashboard} pengguna.
    \item Melakukan manual \textit{testing} terhadap HRIS.
\end{enumerate}

\section{Uraian Pelaksanaan Magang}

\begin{table}
	\centering
	\caption{ Pekerjaan yang dilakukan tiap minggu selama pelaksanaan kerja magang}
	\label{tbl_uraian}
	\begin{tabular}{|c | p{0.75\textwidth}| }
		\hline
		Minggu Ke - & Pekerjaan yang dilakukan \\
		\hline
		1 & \textit{Onboarding} dan mempelajari \textit{flow} sistem HRIS. \\
		\hline
 		2 & Melalukan \textit{research} terkait integrasi modul \textit{Contract} pada HRIS\\
		\hline
 		3 & Mengerjakan UI modul \textit{Contract}. \\
 		\hline
 		4 & Melakukan integrasi API modul \textit{Contract}.\\
 		\hline
 		5 & Melakukan manual \textit{testing} pada modul \textit{Contract}. \\
 		\hline 
 		6 & Mengerjakan UI modul \textit{Payroll System}. \\
 		\hline
            7 & Melakukan integrasi API modul \textit{Payroll System} dan pada \textit{Contract Input}. \\
 		\hline
            8 & Melakukan manual \textit{testing} pada modul \textit{Payroll System}.\\
 		\hline
            9 &  Melakukan perbaikan bug (\textit{bug fixing}) dan proses
        \textit{deployment}.\\
 		\hline
            10 & Melakukan \textit{research} terkait proses \textit{attendance} pada HRIS\\
 		\hline
        11 & Melakukan integrasi API fitur \textit{Attendance}\\
        \hline
        12 & Mengerjakan UI fitur \textit{Lunch Attendance}\\
        \hline
        13 & Melakukan integrasi \textit{Attendance} dengan \textit{Contract Schedule}\\
        \hline
        14 & Melakukan manual \textit{testing} pada fitur \textit{Attendance}\\
        \hline
        15 & Melakukan pengembangan UI modul \textit{Event}\\
        \hline
        16 & Melakukan integrasi API modul \textit{Event}\\
        \hline
        17 & Melakukan manual \textit{testing} pada modul \textit{Event}\\
        \hline
        18 & Melakukan perbaikan bug (\textit{bug fixing}) dan proses \textit{deployment}\\
        \hline
	\end{tabular}
\end{table}

\hspace{2em} Tabel 3.1 di atas merupakan uraian pelaksanaan magang yang dilaksanakan di PT Visi Karya Nusantara. Selama 18 minggu, terlibat dalam pengembangan \textit{frontend} dimulai dari proses \textit{onboarding} dan mempelajari sistem HR. Tahap awal mencakup pembuatan modul \textit{Contract}, melakukan integrasi API, serta melakukan manual \textit{testing} untuk memastikan fungsionalitas berjalan dengan baik. Selanjutnya, dilakukan pengembangan modul \textit{Payroll System} yang kemudian diintegrasikan dengan modul \textit{Contract}, dilanjutkan dengan proses \textit{manual testing} untuk memastikan kedua modul bekerja secara selaras.

\hspace{2em} Pada tahap berikutnya, dikembangkan sistem \textit{Attendance} yang terhubung dengan modul \textit{Contract}, termasuk penambahan fitur \textit{Lunch Attendance}. Setelah itu, dilakukan pengembangan modul \textit{Event} beserta fitur-fiturnya, serta integrasi API dan proses \textit{manual testing}. Tahap akhir meliputi pengecekan keseluruhan halaman dan fitur, melakukan \textit{bug fixing}, dan menjalankan proses \textit{deployment} ke \textit{production} hingga sistem siap digunakan sebagai produk akhir.

\section{Perancangan}

\hspace{2em} Sub bab ini menjelaskan proses perancangan aplikasi \textit{Human Resource Information System} (HRIS), yang mencakup \textit{User Requirement, Sitemap,} dan \textit{Flowchart}.
\\

\subsection{\textit{User Requirement}}

\hspace{2em} Sebelum memulai pengembangan, \textit{Software Engineer Manager} memberikan arahan terkait tugas yang akan dilakukan dalam pengembangan \textit{Human Resource Information System} (HRIS). Aspek yang perlu dikembangkan di antaranya adalah sebagai berikut:

\begin{enumerate}
    \item Mengembangkan halaman \textit{Contract} untuk pengelolaan kontrak karyawan, meliputi pembuatan, pengeditan, dan penghapusan kontrak. Selain itu, tersedia fitur \textit{approval}, \textit{rejection}, \textit{cancel}, dan \textit{termination} sesuai dengan status kontrak. Karyawan juga dapat melihat detail kontrak mereka melalui halaman \textit{Contract User}.

    \pagebreak
    \item Mengembangkan halaman \textit{Payroll Setting} untuk pengaturan item gaji dan tunjangan yang digunakan pada \textit{Contract Input}. Halaman ini terdiri dari \textit{Payroll Setting List} yang menampilkan seluruh item payroll yang tersedia, serta \textit{Payroll Setting Input} untuk menambahkan item baru.

    \item Mengembangkan fitur \textit{Attendance System} pada halaman \textit{Dashboard} untuk mengatur proses absensi karyawan. Pembaruan dilakukan dengan menambahkan fitur \textit{clock in} bagi karyawan dengan sistem kerja \textit{shift} maupun \textit{fixed hour}, yang disesuaikan berdasarkan jadwal dan jam kerja pada kontrak masing-masing.

    \item Mengembangkan modul \textit{Event} yang mencakup pembuatan kegiatan dan pengumuman (\textit{announce}) secara langsung pada \textit{dashboard} pengguna. Fitur ini memungkinkan penambahan deskripsi kegiatan, dokumen pendukung, serta pembaruan data secara \textit{real-time} agar pengguna dapat dengan mudah memantau dan mengelola jadwal kegiatan yang akan datang.
    
\end{enumerate}

\subsection{\textit{Sitemap}}

\hspace{2em} Perancangan struktur navigasi melalui \textit{sitemap} sangat penting dalam pengembangan \textit{Human Resource Information System} (HRIS) untuk memastikan alur kerja pengguna tetap terorganisir dan efisien. \textit{Sitemap} ini berfungsi sebagai peta visual yang menggambarkan hierarki informasi serta hubungan antarhalaman.

\begin{figure}
    \centering
    \fbox{\includegraphics[width=0.9 \linewidth]{assets/pics/sitemap hi A.png}}
    \caption{\textit{Sitemap A Human Resource Information System}}
\end{figure}

\hspace{2em} Gambar 3.2 menunjukkan \textit{sitemap} bagian pertama yang mencakup alur autentikasi serta beberapa manajemen data inti dalam sistem HRIS. Proses operasional dimulai dari entitas HRIS menuju halaman \textit{login}, di mana setiap pengguna diwajibkan melakukan proses autentikasi identitas untuk keamanan akses data perusahaan. Setelah proses \textit{login} berhasil, sistem akan mengarahkan pengguna ke halaman \textit{dashboard} yang merupakan halaman utama untuk mengakses berbagai modul administratif seperti \textit{User Management, Role Management,} hingga \textit{People Report Page}.

\hspace{2em} Sebagai kelanjutan dari struktur sebelumnya, bagian kedua dari \textit{sitemap} ini berfokus pada modul-modul operasional yang lebih spesifik dan kompleks dalam mendukung kegiatan harian perusahaan. Bagian ini mencakup fungsionalitas pengelolaan keuangan, absensi, hingga pengaturan agenda internal tim. Diagram sitemap ini memvisualisasi hubungan antarhalaman fitur-fitur baru yang dikembangkan selama masa magang.

\begin{figure}
    \centering
    \fbox{\includegraphics[width=1 \linewidth]{assets/pics/sitemap hi B.png}}
    \caption{\textit{Sitemap B Human Resource Information System}}
\end{figure}

\hspace{2em} Gambar 3.3 memvisualisasikan kelanjutan navigasi sistem yang mencakup modul-modul utama seperti \textit{Reimbursement, Event, Daily Attendance,} dan \textit{Payroll}. Fokus utama pengembangan dalam laporan magang ini meliputi modul yang diberikan tanda \textit{highlight}, yaitu modul \textit{Dashboard, Leave Setting Input, Contract, Event}, serta \textit{Payroll Setting}. Struktur navigasi ini dirancang untuk memudahkan pengguna dalam mengakses fitur yang dibutuhkan secara terorganisir dan efisien.
\pagebreak
\subsection{\textit{Flowchart}}

\hspace{2em} Pada bagian ini ditampilkan \textit{flowchart} yang menggambarkan alur kerja setiap modul dalam sistem secara menyeluruh. Diagram ini berfungsi untuk memberikan pemahaman yang lebih jelas mengenai tahapan proses, keputusan, serta interaksi antar komponen yang terjadi selama sistem berjalan. Dengan adanya visualisasi ini, pembaca dapat melihat bagaimana data diproses, bagaimana logika sistem bekerja, serta bagaimana setiap fitur saling terhubung satu sama lain. Alur lengkap dari proses yang berlangsung pada modul tersebut dapat dilihat sebagai berikut:

\begin{figure}
    \centering
    \fbox{\includegraphics[width=0.7\linewidth]{assets/pics/contract flow.png}}
    \caption{\textit{Flowchart Contract Permission Page}}
\end{figure}

\hspace{2em} Gambar 3.4 menunjukkan \textit{flowchart} pembagian halaman berdasarkan \textit{permission role} yang diberikan pada halaman \textit{Contract}. Pada proses ini, sistem terlebih dahulu melakukan pengecekan terhadap akun pengguna yang sedang melakukan \textit{login} untuk mengidentifikasi \textit{role} yang dimiliki. Berdasarkan \textit{role} tersebut, sistem kemudian menentukan izin (\textit{permission}) yang sesuai, sehingga hanya fitur atau halaman tertentu yang dapat diakses oleh pengguna. Mekanisme ini diterapkan untuk menjaga keamanan data kontrak serta memastikan setiap pengguna hanya dapat mengakses fungsi sesuai dengan tanggung jawabnya.

\begin{figure}
    \centering
    \fbox{\includegraphics[width=0.7\linewidth]{assets/pics/contract list.png}}
    \caption{\textit{Flowchart Contract List}}
\end{figure}

\hspace{2em} Gambar 3.5 memperlihatkan \textit{flowchart} pada halaman \textit{Contract List}. Proses dimulai dengan pemanggilan \textit{GET API} untuk menampilkan seluruh data \textit{contract} yang telah tersimpan di dalam sistem. Selanjutnya, pengguna yang memiliki \textit{permission create} dapat melakukan pembuatan \textit{contract} baru, sedangkan pengguna dengan \textit{permission approval} dapat melakukan proses persetujuan (\textit{approval}) terhadap \textit{contract} sesuai dengan peran yang dimiliki.

\begin{figure}
    \centering
    \fbox{\includegraphics[width=1\linewidth]{assets/pics/contract create.png}}
    \caption{\textit{Flowchart Contract Create}}
\end{figure}

\hspace{2em} Gambar 3.6 merupakan \textit{flowchart} pada halaman \textit{Contract Create}. Proses dimulai dengan menampilkan \textit{form field} yang digunakan untuk mengisi data perjanjian karyawan. Pada halaman ini, pengguna dapat melakukan kustomisasi terhadap data \textit{penalty \& rule}, jadwal kerja karyawan, serta komponen gaji yang disesuaikan dengan kebutuhan. Setelah seluruh data diisi, sistem akan menampilkan \textit{modal} konfirmasi, dan setelah pengguna memberikan persetujuan, data kontrak akan disimpan ke dalam \textit{database}.

\begin{figure}
    \centering
    \fbox{\includegraphics[width=1\linewidth]{assets/pics/contract detail.png}}
    \caption{\textit{Flowchart Contract Detail Approval}}
\end{figure}

\pagebreak
\hspace{2em} Gambar 3.7 menunjukkan \textit{flowchart} pada halaman \textit{Contract Detail Approval}. Proses dimulai dengan pemanggilan GET API untuk mengambil data kontrak yang dipilih. Jika pengguna memiliki \textit{permission approval}, maka dapat melakukan \textit{accept} atau \textit{reject} terhadap kontrak. Setelah itu, ditampilkan modal konfirmasi, dan kontrak disimpan ke \textit{database} serta dikirim kepada pengguna untuk mengaktifkan masa kontrak.

\begin{figure}
    \centering
    \fbox{\includegraphics[width=0.7\linewidth]{assets/pics/payroll list.png}}
    \caption{\textit{Flowchart Payroll List}}
\end{figure}

\hspace{2em} Gambar 3.8 adalah \textit{flowchart} pada halaman \textit{Payroll List}. Proses dimulai dengan pemanggilan GET API untuk menampilkan seluruh item \textit{Payroll Setting}. Pengguna dapat melihat seluruh item Payroll Setting apabila mendapatkan \textit{permission read}. Pengguna dapat membuat item \textit{payroll} baru apabila mendapatkan \textit{permission create} berdasarkan role yang dimiliki.

\begin{figure}
    \centering
    \fbox{\includegraphics[width=1\linewidth]{assets/pics/payroll create.png}}
    \caption{\textit{Flowchart Payroll Create}}
\end{figure}

\hspace{2em} Gambar 3.9 merupakan \textit{flowchart} pada halaman \textit{Payroll Create}. Proses dimulai dengan menampilkan \textit{form field} untuk pengisian \textit{item payroll} oleh \textit{admin}, di mana data yang diperlukan mencakup \textit{item payroll} dan \textit{description}. Setelah \textit{admin} melengkapi data, sistem akan memunculkan sebuah modal konfirmasi pengajuan untuk melakukan validasi. Jika konfirmasi disetujui, maka data akan disimpan ke dalam \textit{database} agar dapat digunakan kembali secara fleksibel dalam modul \textit{payslip} dan \textit{contract} karyawan.

\begin{figure}
    \centering
    \fbox{\includegraphics[width=0.65\linewidth]{assets/pics/dashboard flow.png}}
    \caption{\textit{Flowchart Attendance System}}
\end{figure}

    \hspace{2em} Gambar 3.10 memperlihatkan \textit{flowchart} pada halaman \textit{Dashboard Permission}. Apabila pengguna merupakan \textit{superadmin}, maka sistem akan menampilkan \textit{dashboard overview} yang berisi data-data internal. Jika bukan, sistem akan memeriksa apakah pengguna memiliki \textit{work entry} ``\textit{shift}'' atau tidak. Jika iya, maka hanya fitur \textit{attendance system} yang akan ditampilkan, sedangkan jika tidak, akan terdapat alur tambahan untuk \textit{lunch system}.


\begin{figure}
    \centering
    \fbox{\includegraphics[width=0.9\linewidth]{assets/pics/dashboard fixed.png}}
    \caption{\textit{Flowchart Attendance System Fixed Type}}
\end{figure}

\hspace{2em} Gambar 3.11 memperlihatkan \textit{flowchart} pada sistem \textit{attendance} dengan jenis \textit{work entry} \textit{fixed}. Proses dimulai dengan pengecekan apakah pengguna sudah melakukan \textit{clock-in}. Jika belum, sistem akan memeriksa keterlambatan. Apabila pengguna terlambat, mereka wajib mengisi alasan keterlambatan sebelum data disimpan ke dalam \textit{database}. Setelah \textit{clock-in}, sistem akan menampilkan fitur \textit{lunch system} dan memeriksa apakah pengguna sudah melakukan \textit{lunch in}. Jika belum, pengguna dapat melakukan \textit{lunch in}, kemudian \textit{lunch end}. Apabila \textit{lunch end} dilakukan terlambat, pengguna juga harus mengisi alasannya. Jika pengguna sudah melakukan \textit{clock-in} dan \textit{lunch in}, sistem akan memeriksa apakah pengguna mencoba \textit{clock-out} terlalu awal maka harus mengisi alasannya. Lalu, sistem akan menyimpan data ke dalam \textit{database}.

\begin{figure}
    \centering
    \fbox{\includegraphics[width=1\linewidth]{assets/pics/dashboard shift.png}}
    \caption{\textit{Flowchart Attendance System Shift Type}}
\end{figure}

\hspace{2em} Gambar 3.12 memperlihatkan \textit{flowchart} pada sistem \textit{attendance} dengan jenis \textit{work entry} \textit{shift}. Proses dimulai dengan pengecekan apakah pengguna sudah melakukan \textit{clock-in}. Jika belum, sistem akan meminta jadwal \textit{shift} dan memeriksa keterlambatan. Apabila pengguna terlambat, mereka wajib mengisi alasan keterlambatan sebelum data disimpan ke dalam \textit{database}. Sebaliknya, jika pengguna sudah melakukan \textit{clock-in}, sistem akan memeriksa apakah pengguna mencoba \textit{clock-out} lebih awal. Jika iya, pengguna diwajibkan mengisi alasan sebelum data disimpan ke dalam \textit{database}.

\begin{figure}
    \centering
    \fbox{\includegraphics[width=0.9\linewidth]{assets/pics/event flow.png}}
    \caption{\textit{Flowchart Event Calendar}}
\end{figure}

\hspace{2em} Gambar 3.13 menunjukkan \textit{flowchart} pada halaman \textit{Event Calendar}. Proses diawali dengan pemanggilan GET API untuk mengambil data \textit{event} dan hari libur yang telah tersimpan. Pengguna dapat melihat detail \textit{event} melalui modal yang menampilkan informasi lengkap beserta opsi untuk mengedit atau menghapus \textit{event}. Jika pengguna memilih untuk menghapus, sistem akan menampilkan modal konfirmasi terlebih dahulu, dan hanya pembuat \textit{event} yang memiliki hak untuk melakukan penghapusan. Selain itu, pengguna yang memiliki \textit{permission create} dapat membuat \textit{event} baru.

\begin{figure}
    \centering
    \fbox{\includegraphics[width=1\linewidth]{assets/pics/event create.png}}
    \caption{\textit{Flowchart Event Create}}
\end{figure}

\hspace{2em} Gambar 3.14 memperlihatkan \textit{flowchart} pada halaman \textit{Create Event}. Proses dimulai dengan menampilkan \textit{form field} untuk mengisi data \textit{event}. Terdapat pilihan untuk \textit{event type}, seperti \textit{meeting} dan \textit{holiday}. Apabila pengguna memilih \textit{meeting}, maka data yang perlu diisi meliputi nama \textit{meeting}, tanggal, waktu pelaksanaan, daftar karyawan yang terlibat, warna yang akan ditampilkan pada kalender, deskripsi, dan dokumen pendukung. Sementara itu, jika memilih \textit{holiday,} pengguna hanya perlu mengisi tanggal, judul hari libur, serta \textit{optional field} berupa deskripsi dan dokumen pendukung. Setelah pengisian data selesai, akan ditampilkan modal konfirmasi. Apabila disetujui, data \textit{event} tersebut akan dikirimkan dan disimpan ke dalam \textit{database}.

\begin{figure}
    \centering
    \fbox{\includegraphics[width=1\linewidth]{assets/pics/event edit.png}}
    \caption{\textit{Flowchart Event Edit}}
\end{figure}

\hspace{2em} Gambar 3.15 memperlihatkan \textit{flowchart} pada halaman \textit{Edit Event}. Proses dimulai dengan pemanggilan GET API untuk mengambil data \textit{event} yang dipilih untuk diubah. Pengguna dapat melakukan perubahan pada nama \textit{event}, waktu pelaksanaan, daftar karyawan yang terlibat, warna \textit{event}, deskripsi, dan dokumen pendukung. Setelah seluruh perubahan selesai dilakukan, akan ditampilkan modal konfirmasi. Apabila konfirmasi disetujui, data \textit{event} yang telah diperbarui akan dikirimkan dan disimpan ke dalam \textit{database}.

\pagebreak

\section{\textit{Wireframe}}

\hspace{2em} Bagian ini menampilkan \textit{wireframe} sebagai rancangan dasar antarmuka \textit{Human Resource Internal System}. \textit{Wireframe} ini digunakan untuk memvisualisasikan struktur halaman, susunan elemen, serta alur interaksi pengguna sebelum sistem diimplementasikan secara penuh. Dengan adanya \textit{wireframe}, proses perancangan dan pengembangan dapat dilakukan secara lebih terarah serta memastikan bahwa kebutuhan fungsional telah dipetakan dengan baik sejak tahap awal. Rancangan \textit{wireframe} tersebut disajikan sebagai berikut:

\begin{figure}
    \centering
    \fbox{\includegraphics[width=0.8\linewidth]{assets/pics/wire contract list.png}}
    \caption{\textit{Wireframe Contract List}}
\end{figure}

\hspace{2em} Gambar 3.16 adalah \textit{wireframe} pada halaman \textit{Contract List}. Pada halaman ini menampilkan tabel yang berisi seluruh daftar kontrak yang telah dibuat. Beberapa kolom yang ditampilkan meliputi \textit{Contract Name, Employee Name, Start Date, End Date, Job Title, Employee Status, Created At, Created By, Status,} dan \textit{Action}. Selain itu, tersedia tombol untuk membuat kontrak baru yang akan mengarahkan pengguna ke halaman \textit{Contract Input}. Pada bagian \textit{Action}, terdapat ikon detail (\textit{eye}) yang digunakan untuk memperbarui status atau melihat informasi lengkap kontrak, serta ikon \textit{edit} (\textit{pencil}) yang memungkinkan pengubahan data kontrak selama status kontrak masih \textit{pending}.

\begin{figure}
    \centering
    \fbox{\includegraphics[width=0.8\linewidth]{assets/pics/wire contract create.png}}
    \caption{\textit{Wireframe Contract Input}}
\end{figure}

\hspace{2em} Gambar 3.17 adalah \textit{wireframe} pada halaman \textit{Contract Input}. Halaman ini digunakan untuk membuat kontrak baru bagi karyawan. Terdapat beberapa data yang harus dipenuhi yaitu pada bagian \textit{General Information} terdapat \textit{form fields employee name, contract name, job title, employee status, supervisor, work location, start date, end date,} dan \textit{leave}. Bagian \textit{Work Schedule} memuat \textit{form field} untuk \textit{work entry} dan \textit{work schedule}. Selanjutnya, pada bagian \textit{Attendance and Penalty Rule}, terdapat pengaturan \textit{Missing In Action} serta berbagai penalti lain yang dapat disesuaikan dengan kebijakan perusahaan. Terakhir, bagian \textit{Salary and Benefit} berisi informasi mengenai gaji pokok serta tambahan tunjangan atau manfaat lainnya.

\begin{figure}
    \centering
    \fbox{\includegraphics[width=0.8\linewidth]{assets/pics/wire contract detail.png}}
    \caption{\textit{Wireframe Contract Detail}}
\end{figure}

\hspace{2em} Gambar 3.18 adalah \textit{wireframe} pada halaman \textit{Contract Detail}. Halaman ini menampilkan seluruh informasi kontrak yang telah dibuat dan menyediakan sejumlah aksi yang dapat dilakukan, seperti \textit{approve, reject, cancel,} atau \textit{terminate}, sesuai dengan status kontrak yang sedang berjalan. Apabila status kontrak adalah \textit{pending} maka aksi yang dapat dilakukan \textit{accept} atau \textit{reject}. Jika status kontrak adalah \textit{approve} maka aksi yang dapat dilakukan adalah \textit{cancel} dan status kontrak \textit{on going} maka aksi yang dapat dilakukan adalah \textit{terminate}.

\begin{figure}
    \centering
    \fbox{\includegraphics[width=1\linewidth]{assets/pics/wire payroll list.png}}
    \caption{\textit{Wireframe Payroll setting List}}
\end{figure}

\hspace{2em} Gambar 3.19 adalah \textit{wireframe} pada halaman \textit{Payroll setting List}. Halaman ini menampilkan seluruh daftar \textit{item payroll} yang telah dibuat dan akan digunakan pada bagian\textit{ Salary and Benefit} saat proses pembuatan kontrak. Beberapa kolom yang ditampilkan meliputi \textit{Item Name, Description, Date Created,} dan \textit{Action}. Selain itu, tersedia tombol untuk menambahkan \textit{item payroll} baru yang akan mengarahkan pengguna ke halaman \textit{Payroll Setting Input}.

\begin{figure}
    \centering
    \fbox{\includegraphics[width=1\linewidth]{assets/pics/wire payroll create.png}}
    \caption{\textit{Wireframe Payroll setting Input}}
\end{figure}

\hspace{2em} Gambar 3.20 adalah \textit{wireframe} pada halaman \textit{Payroll setting Input}. Halaman ini digunakan untuk menambahkan \textit{item payroll} baru, dan memungkinkan penambahan lebih dari satu data sekaligus melalui tombol \textit{Add Row}. Terdapat \textit{form fields} yang perlu diisi, yaitu \textit{item name} dan \textit{description}, sebelum item tersebut dapat disimpan ke dalam daftar \textit{payroll}.

\begin{figure}
    \centering
    \fbox{\includegraphics[width=1\linewidth]{assets/pics/wire payroll detail.png}}
    \caption{\textit{Wireframe Payroll setting Detail}}
\end{figure}

\hspace{2em} Gambar 3.21 adalah \textit{wireframe} pada halaman \textit{Payroll setting Detail}. Halaman ini menampilkan informasi lengkap mengenai \textit{item payroll} yang telah dibuat, seperti \textit{item name} dan \textit{description}. Selain menampilkan informasi, halaman ini dapat untuk memastikan bahwa setiap komponen gaji yang didefinisikan sudah memiliki penjelasan yang cukup.

\begin{figure}
    \centering
    \fbox{\includegraphics[width=1\linewidth]{assets/pics/wire attend fixed.png}}
    \caption{\textit{Wireframe Attendance work type Fixed}}
\end{figure}

\hspace{2em} Gambar 3.22 adalah \textit{wireframe} pada halaman \textit{Dashboard attendance work type fixed}. Pada halaman ini, pengguna dapat melakukan proses absensi yang meliputi \textit{clock in, clock out}, serta absensi untuk jam makan siang sesuai dengan jadwal yang telah ditetapkan pada kontrak kerja dengan tipe \textit{fixed}. Apabila pengguna tidak mendapatkan jam makan siang, maka fitur tersebut tidak akan ditampilkan pada halaman ini.

\hspace{2em} Pengguna memulai proses absensi dengan melakukan \textit{clock in} sesuai jadwal yang telah ditetapkan. Setelah itu, pengguna mengisi aktivitas kerja yang akan ditampilkan pada bagian \textit{standup feed}. Selanjutnya, apabila pengguna memiliki jadwal makan siang, sistem \textit{lunch} akan aktif dan menjalankan \textit{countdown} hingga waktu makan siang berakhir. Namun, bagi pengguna yang tidak memiliki jadwal makan siang, proses absensi tetap dapat dilanjutkan dengan melakukan \textit{clock out} sesuai dengan ketentuan yang tercantum pada kontrak kerja.
\begin{figure}
    \centering
    \fbox{\includegraphics[width=0.9\linewidth]{assets/pics/wire attend shift.png}}
    \caption{\textit{Wireframe Attendance work type Shift}}
\end{figure}

\hspace{2em} Gambar 3.23 adalah \textit{wireframe} pada halaman \textit{Dashboard Attendance Work Type Shift}. Pada halaman ini, pengguna dapat melakukan proses absensi dengan memilih jadwal \textit{shift} yang akan dijalani terlebih dahulu. Pilihan \textit{shift} tersebut ditampilkan berdasarkan jadwal kerja yang telah ditetapkan pada kontrak kerja dengan tipe \textit{shift}. Melalui tampilan ini, sistem memastikan bahwa proses absensi dilakukan sesuai dengan jadwal yang berlaku sehingga data kehadiran karyawan dapat tercatat secara akurat.

\begin{figure}
    \centering
    \fbox{\includegraphics[width=0.9\linewidth]{assets/pics/wire event dash.png}}
    \caption{\textit{Wireframe Event Dashboard}}
\end{figure}

\hspace{2em} Gambar 3.24 adalah \textit{wireframe} pada halaman \textit{Dashboard}. Pada halaman ini menampilkan daftar \textit{event} yang tersedia sebagai \textit{announcement} pada halaman utama, serta memungkinkan pengguna untuk melihat \textit{event} berdasarkan tanggal yang dipilih. Detail informasi \textit{event} dapat diakses melalui list yang tersedia dengan mengklik judul \textit{event} tersebut. Fitur \textit{event} ini tersedia di semua halaman \textit{dashboard} pengguna yang memiliki akses kedalam \textit{Human Resource Information System} (HRIS) .

\begin{figure}
    \centering
    \fbox{\includegraphics[width=1\linewidth]{assets/pics/wire calendar view.png}}
    \caption{\textit{Wireframe Event Calendar}}
\end{figure}

\hspace{2em} Gambar 3.25 adalah \textit{wireframe} pada halaman \textit{Event Calendar}. Pada halaman ini menampilkan daftar jadwal dalam bentuk kalender dan memungkinkan pengguna untuk melihat tampilan jadwal berdasarkan mode \textit{month, week,} maupun \textit{day}, sehingga detail tanggal dan waktu kegiatan dapat dipantau dengan lebih mudah dan terstruktur. Terdapat list kegiatan di setiap tanggalnya yang menampilkan maksimal tiga kegiatan. Untuk menambahkan kegiatan baru, pengguna dapat mengklik tombol (\textit{add event}) yang akan mengarahkan ke halaman \textit{Event Input}.

\begin{figure}
    \centering
    \fbox{\includegraphics[width=0.9\linewidth]{assets/pics/wire calendar create.png}}
    \caption{\textit{Wireframe Event Input}}
\end{figure}

\hspace{2em} Gambar 3.26 adalah \textit{wireframe} pada halaman \textit{Event Input}. Pada halaman ini digunakan untuk menambahkan \textit{event} baru dengan memilih jenis \textit{event} terlebih dahulu, seperti \textit{meeting, holiday,} atau \textit{schedule}. Setelah menentukan jenisnya, pengguna dapat mengisi sejumlah \textit{form field} yang tersedia, antara lain \textit{event name, date, start time, end time, color, guest, description,} serta mengunggah dokumen pendukung sebagai kelengkapan informasi terkait kegiatan yang akan dilaksanakan.

\begin{figure}
    \centering
    \fbox{\includegraphics[width=1\linewidth]{assets/pics/wire calendar modal.png}}
    \caption{\textit{Wireframe Event Modal}}
\end{figure}

\hspace{2em} Gambar 3.27 adalah \textit{wireframe} pada bagian \textit{Event Modal} pada halaman \textit{Event Calendar}. Bagian ini menampilkan detail lengkap dari jadwal yang dipilih, meliputi judul kegiatan, deskripsi, media atau \textit{file} terkait, daftar tamu yang diundang, serta tipe jadwal yang sedang ditampilkan. File tersebut dapat diakses langsung melalui tautan yang tersedia. Selain itu, terdapat opsi untuk mengedit atau menghapus jadwal, di mana hanya pembuat jadwal yang memiliki hak akses untuk melakukan penghapusan.

\section{Implementasi}

\hspace{2em} \textit{Human Resource Internal System} telah berhasil dikembangkan dan diimplementasikan pada mode \textit{production}, serta resmi dijadikan produk yang digunakan oleh klien terkait. Berikut ini merupakan tampilan akhir dari \textit{Human Resource Internal System} yang telah selesai dikembangkan:

\begin{figure}
    \centering
    \fbox{\includegraphics[width=0.9\linewidth]{assets/pics/contract list view.png}}
    \caption{Tampilan \textit{Contract List}}
\end{figure}

\hspace{2em} Gambar 3.28 menunjukkan tampilan halaman \textit{Contract List}. Pada halaman ini terdapat tabel yang menampilkan daftar kontrak yang telah dibuat. Beberapa kolom yang ditampilkan meliputi \textit{Contract Name, Employee Name, Start Date, End Date, Job Title, Employee Status, Created At, Created By, Status,} dan \textit{Action}. Jika \textit{user} ingin membuat kontrak baru, tersedia tombol yang mengarahkan ke halaman \textit{Contract Input}. Untuk kontrak yang berstatus pending, terdapat aksi \textit{Edit} (ikon \textit{pencil}) yang membawa \textit{user} ke halaman \textit{edit} kontrak untuk melakukan pembaruan atau penghapusan data. Selain itu, tersedia ikon detail (\textit{eye}) yang digunakan untuk memperbarui status kontrak atau melihat informasi lengkap dari kontrak tersebut.

\begin{figure}
    \centering
    \fbox{\includegraphics[width=1\linewidth]{assets/pics/contract input.png}}
    \caption{Tampilan \textit{Contract Input General Information} dan \textit{Work Schedule}}
\end{figure}

\hspace{2em} Gambar 3.29 menunjukkan tampilan halaman \textit{Contract Input}. Pada halaman ini, user dapat membuat kontrak baru dengan mengisi sejumlah \textit{form field} yang telah disediakan. Pada bagian \textit{General Information}, terdapat \textit{form field} seperti \textit{employee name, contract name, job title, employee status, supervisor, work location, start date, end date,} dan \textit{leave}. Sementara itu, bagian \textit{Work Schedule} mencakup \textit{form field} untuk \textit{work entry} dan \textit{work schedule}, di mana opsi pada \textit{work schedule} akan menyesuaikan secara otomatis berdasarkan pilihan \textit{work entry} yang dipilih, yaitu antara tipe \textit{fixed} atau \textit{shift}.

\begin{figure}
    \centering
    \fbox{\includegraphics[width=1\linewidth]{assets/pics/contract input 2.png}}
    \caption{Tampilan \textit{Contract Input Attendance and Penalty Rule} dan \textit{Salary and Benefit}}
\end{figure}

\hspace{2em} Gambar 3.30 menunjukkan tampilan halaman \textit{Contract Input}. Pada halaman ini, user dapat melanjtukan dalam pembuatan kontrak baru dengan mengisi sejumlah \textit{form field} yang telah disediakan. Pada bagian \textit{Attendance and Penalty Rule}, terdapat pengaturan mengenai \textit{Missing In Action} yang wajib di isi serta berbagai jenis penalti lain yang dapat disesuaikan dengan kebijakan perusahaan. Terakhir, bagian \textit{Salary and Benefit} memuat informasi terkait gaji pokok yang diharuskan di isi dengan \textit{unit addition} serta tambahan tunjangan atau manfaat lainnya.

\begin{figure}
    \centering
    \fbox{\includegraphics[width=1\linewidth]{assets/pics/add job title.png}}
    \caption{Tampilan Fitur \textit{Add Job Title} pada \textit{Contract Input}}
\end{figure}

\hspace{2em} Gambar 3.31 menunjukkan tampilan fitur \textit{Add Job Title} pada halaman \textit{Contract Input}. Fitur ini memungkinkan \textit{user} untuk menambahkan \textit{job title} baru secara langsung tanpa harus meninggalkan halaman pembuatan kontrak. Dengan mengklik ikon tambah (\textit{plus}) di dalam \textit{option form field job title}, akan muncul sebuah modal yang berisi \textit{form field} untuk mengisi nama \textit{job title} baru. Setelah data diisi, pengguna dapat menyimpan perubahan tersebut, dan \textit{job title} baru akan langsung tersedia untuk dipilih pada halaman \textit{Contract Input}.

\begin{figure}
    \centering
    \fbox{\includegraphics[width=1\linewidth]{assets/pics/leave contract.png}}
    \caption{Tampilan Fitur \textit{Leave} pada \textit{Contract Input}}
\end{figure}

\hspace{2em} Gambar 3.32 menunjukkan tampilan fitur \textit{Leave} pada halaman \textit{Contract Input}. Fitur ini berfungsi untuk menampilkan jumlah cuti yang berlaku bagi karyawan sesuai ketentuan yang telah ditetapkan pada \textit{Leave Setting}. Sistem akan melakukan penyaringan (\textit{filter}) secara otomatis berdasarkan \textit{job title} dan \textit{employee status}, sehingga hak cuti yang ditampilkan sudah sesuai dengan kebijakan perusahaan untuk karyawan tersebut.

\begin{figure}
    \centering
    \fbox{\includegraphics[width=1\linewidth]{assets/pics/leave setting input.png}}
    \caption{Tampilan \textit{Leave Setting Input}}
\end{figure}

\pagebreak
\hspace{2em} Gambar 3.33 menunjukkan tampilan halaman \textit{Leave Setting Input}. Pada halaman ini, \textit{user} dapat membuat pengaturan cuti baru yang nantinya akan digunakan pada bagian \textit{Leave} saat proses pembuatan kontrak. Tersedia beberapa \textit{form field} yang dapat diisi, yaitu \textit{Leave Name, Job Title, Employee Status, Leave Duration}, dan \textit{Leave Interval}. Pengaturan cuti ini dapat disesuaikan (\textit{customized}) berdasarkan \textit{job title, employee status,} atau kombinasi keduanya, sehingga hak cuti dapat diatur secara lebih spesifik sesuai kebutuhan perusahaan dan dapat difilter secara otomatis ketika kontrak karyawan dibuat.

\begin{figure}
    \centering
    \fbox{\includegraphics[width=1\linewidth]{assets/pics/contract detail view.png}}
    \caption{Tampilan \textit{Contract Detail}}
\end{figure}

\hspace{2em} Gambar 3.34 menunjukkan tampilan halaman \textit{Contract Detail}. Pada halaman ini menampilkan informasi lengkap dari kontrak yang telah dibuat dan menyediakan beberapa opsi tindakan sesuai dengan status kontrak. Jika kontrak berstatus \textit{pending}, tersedia pilihan \textit{approve} dan \textit{reject}. Apabila kontrak berstatus \textit{approved}, pengguna dapat melakukan \textit{cancel contract}. Sementara itu, untuk kontrak dengan status \textit{on-going}, tersedia opsi \textit{terminate}. Jika kontrak berstatus \textit{rejected}, maka kontrak tersebut dianggap tidak berlaku dan harus dibuat ulang melalui pembuatan kontrak baru.

\begin{figure}
    \centering
    \fbox{\includegraphics[width=0.8\linewidth]{assets/pics/payroll setting list.png}}
    \caption{Tampilan \textit{Payroll setting List}}
\end{figure}

\hspace{2em} Gambar 3.35 menunjukkan tampilan halaman \textit{Payroll Setting List}. Halaman ini menampilkan daftar \textit{item payroll} yang telah dibuat, dengan beberapa kolom seperti \textit{Item Name, Description, Date Created,} dan \textit{Action}. Tersedia tombol \textit{Add Payroll Item} untuk menambahkan \textit{item payroll} baru. Selain itu, pada setiap item terdapat aksi \textit{Edit} (ikon \textit{pencil}) yang akan mengarahkan pengguna ke halaman \textit{Payroll Edit}, serta aksi \textit{Delete} (ikon tempat sampah). Namun, untuk item Gaji Pokok/\textit{Main Salary}, seluruh aksi tersebut tidak dapat digunakan karena item tersebut bersifat wajib (\textit{required}) dan tidak dapat diubah maupun dihapus.

\begin{figure}
    \centering
    \fbox{\includegraphics[width=0.8\linewidth]{assets/pics/payroll setting delete.png}}
    \caption{Tampilan \textit{Payroll setting List} Modal Delete}
\end{figure}

\hspace{2em} Gambar 3.36 menunjukkan tampilan modal \textit{Delete} pada halaman \textit{Payroll Setting List}. Modal ini berfungsi sebagai konfirmasi sebelum suatu item \textit{payroll} dihapus dari sistem. Dengan adanya modal ini, pengguna dapat memastikan tindakan penghapusan benar-benar diinginkan sehingga mengurangi risiko kesalahan penghapusan data.

\begin{figure}
    \centering
    \fbox{\includegraphics[width=0.9\linewidth]{assets/pics/payroll setting input.png}}
 \caption{Tampilan \textit{Payroll Setting Input}}
 \end{figure}
 
 \hspace{2em} Gambar 3.37 menunjukkan tampilan halaman \textit{Payroll Setting Input}. Pada halaman ini, \textit{user} dapat membuat item \textit{payroll} baru yang nantinya dapat dipilih saat proses pembuatan kontrak pada bagian \textit{salary and benefit}. Tersedia beberapa \textit{form field} utama, yaitu \textit{Item Name} dan \textit{Description}, yang digunakan untuk mendefinisikan detail item \textit{payroll} yang akan ditambahkan ke dalam sistem.
 
 \begin{figure}
    \centering
    \fbox{\includegraphics[width=0.9\linewidth]{assets/pics/payroll setting input 2.png}}
 \caption{Tampilan \textit{Payroll Setting Input Add Row}}
 \end{figure}

\hspace{2em} Gambar 3.38 menampilkan kondisi halaman \textit{Payroll Setting Input} ketika fitur \textit{Add Row} digunakan. Fitur ini memungkinkan \textit{user} untuk menambahkan lebih dari satu item \textit{payroll} dalam satu kali proses input. Dengan adanya fitur ini, proses pengelolaan data \textit{payroll} menjadi lebih efisien karena pengguna tidak perlu melakukan penginputan item satu per satu secara terpisah.

\begin{figure}
    \centering
    \fbox{\includegraphics[width=0.8\linewidth]{assets/pics/attendance fixed.png}}
    \caption{Tampilan \textit{Attendance work type Fixed}}
\end{figure}

\hspace{2em} Gambar 3.39 menunjukkan tampilan halaman \textit{Dashboard}. Pada halaman ini, \textit{user} dapat melakukan sistem \textit{attendance} dengan \textit{work type Fixed}, sehingga akan muncul \textit{card} tambahan \textit{Lunch Attendance} apabila terdapat jadwal \textit{lunch} pada kontrak yang telah dibuat. Sistem \textit{attendance} akan aktif berdasarkan jadwal yang ditetapkan di dalam kontrak, dan fitur \textit{Lunch Time} akan aktif setelah \textit{user} melakukan \textit{Clock In}.

\begin{figure}
    \centering
    \fbox{\includegraphics[width=0.8\linewidth]{assets/pics/attendance shift.png}}
    \caption{Tampilan \textit{Attendance work type Shift}}
\end{figure}

\hspace{2em} Gambar 3.40 menunjukkan tampilan halaman \textit{Dashboard}. Pada halaman ini, \textit{user} dapat melakukan \textit{attendance} dengan \textit{work type Shift}, sehingga tersedia \textit{dropdown} untuk memilih jadwal \textit{shift} sebelum melakukan \textit{Clock In}. Sistem akan menolak proses \textit{Clock In} apabila \textit{user} tidak memilih \textit{shift} yang sesuai dengan waktu masuk yang ditentukan.

\begin{figure}
    \centering
    \fbox{\includegraphics[width=0.8\linewidth]{assets/pics/dashboard before.png}}
    \caption{Tampilan halaman \textit{Dashboard} \textit{Before}}
\end{figure}

\hspace{2em} Gambar 3.41 menunjukkan tampilan \textit{Dashboard} sebelum ditambahkan fitur \textit{Event Announcement}. Pada tampilan sebelumnya, belum tersedia informasi terkait acara yang berlangsung, sehingga menyulitkan \textit{user} untuk mengetahui acara apa saja yang ada pada hari tertentu dan mengharuskan mereka membuka halaman \textit{Event} terlebih dahulu.

\begin{figure}
    \centering
    \fbox{\includegraphics[width=0.8\linewidth]{assets/pics/dashboard after.png}}
    \caption{Tampilan halaman \textit{Dashboard} \textit{After}}
\end{figure}

\hspace{2em} Gambar 3.42 menunjukkan tampilan \textit{Dashboard} setelah penambahan fitur \textit{Event Announcement}. Fitur ini berfungsi untuk menampilkan informasi acara yang diadakan berdasarkan hari yang dipilih, sehingga memudahkan \textit{user} untuk mengetahui acara yang tersedia maupun yang baru ditambahkan tanpa perlu membuka halaman \textit{Event} terlebih dahulu. Selain itu, fitur ini telah dilengkapi dengan implementasi Websocket untuk memberikan pembaruan informasi acara secara \textit{real-time}, sehingga setiap perubahan dapat langsung terlihat oleh pengguna.

\begin{figure}
    \centering
    \fbox{\includegraphics[width=1\linewidth]{assets/pics/event calendar.png}}
    \caption{Tampilan Jadwal \textit{Event}}
\end{figure}

\hspace{2em} Gambar 3.43 menunjukkan tampilan halaman \textit{Event Calendar} yang menampilkan seluruh kegiatan yang telah dibuat berdasarkan tanggal dan waktu yang telah ditetapkan. Pada halaman ini, pengguna dapat melihat daftar kegiatan secara ringkas dalam tampilan kalender. Apabila terdapat lebih dari dua kegiatan pada tanggal yang sama, sistem menyediakan tombol \textit{See More} yang akan mengarahkan pengguna ke tampilan mode \textit{Day} untuk melihat seluruh kegiatan pada hari tersebut secara lebih detail.

\begin{figure}
    \centering
    \fbox{\includegraphics[width=1\linewidth]{assets/pics/event calendar collapse.png}}
    \caption{Tampilan Jadwal \textit{Event collapse}}
\end{figure}

\pagebreak

\hspace{2em} Gambar 3.44 menampilkan kondisi ketika terdapat jadwal kegiatan yang saling bertumpukan atau \textit{collapse} pada waktu yang sama. Pada kondisi ini, sistem akan menampilkan sebuah \textit{modal} yang berisi daftar kegiatan dengan waktu yang saling bertabrakan. Melalui tampilan tersebut, pengguna dapat mengetahui seluruh kegiatan yang berlangsung pada waktu yang sama tanpa kehilangan informasi terkait jadwal lainnya.

\hspace{2em} Selain itu, sistem juga menyediakan indikator notifikasi untuk membantu pengguna memantau kegiatan yang belum ditinjau. Pada menu \textit{Event} di \textit{sidebar}, ditampilkan \textit{badge notification} yang menunjukkan jumlah total \textit{event} yang belum dibuka atau dilihat detailnya. Di dalam kalender, \textit{event} yang belum dibuka ditandai dengan sebuah titik (\textit{dot}) berwarna putih sebagai penanda visual bahwa informasi kegiatan tersebut masih perlu diperhatikan oleh pengguna.

\begin{figure}
    \centering
    \fbox{\includegraphics[width=1\linewidth]{assets/pics/event modal meeting.png}}
    \caption{ Tampilan modal detail \textit{Event type Meeting} atau \textit{Schedule}}
\end{figure}

\hspace{2em} Gambar 3.45 menunjukkan tampilan modal detail \textit{Event Calendar} untuk tipe \textit{meeting} atau \textit{schedule}. Pada bagian ini ditampilkan informasi lengkap mengenai kegiatan, seperti judul \textit{event}, deskripsi, media atau dokumen terkait, tanggal dan waktu pelaksanaan, daftar tamu yang diundang, serta tipe \textit{event}.

\begin{figure}
    \centering
    \fbox{\includegraphics[width=1\linewidth]{assets/pics/event modal holiday.png}}
    \caption{Tampilan modal detail \textit{Event type Holiday}}
\end{figure}

\hspace{2em} Gambar 3.46 menampilkan modal detail \textit{Event Calendar} untuk tipe \textit{holiday}. Modal ini hanya berisi informasi judul hari libur, deskripsi, serta tanggal pelaksanaan. Untuk tipe \textit{holiday}, \textit{event} secara otomatis dikirimkan ke seluruh pengguna dan diatur berlangsung sepanjang hari (\textit{all day}).

\hspace{2em} Selain itu, terdapat tombol aksi \textit{Edit} (ikon \textit{pencil}) yang mengarahkan pengguna ke halaman \textit{}Event Edit untuk memperbarui kegiatan, serta aksi \textit{Delete} (ikon tempat sampah) untuk menghapus \textit{event}. Aksi ini hanya dapat dilakukan oleh pengguna yang membuat kegiatan tersebut, sedangkan pengguna yang diundang hanya dapat melihat detail informasi tanpa dapat melakukan perubahan.

\begin{figure}
    \centering
    \fbox{\includegraphics[width=1\linewidth]{assets/pics/event input before.png}}
    \caption{Tampilan halaman \textit{Event Input} \textit{Before}}
\end{figure}

\hspace{2em} Gambar 3.47 menunjukkan tampilan halaman \textit{Event Input} sebelum dilakukan \textit{revamp}. Pada halaman tersebut, \textit{user} dapat membuat \textit{event}, namun belum tersedia fitur untuk menambahkan \textit{supporting document} atau \textit{description} yang diperlukan untuk memberikan detail tambahan mengenai \textit{event} yang akan dibuat.

\begin{figure}
    \centering
    \fbox{\includegraphics[width=1\linewidth]{assets/pics/event input after.png}}
    \caption{Tampilan halaman \textit{Event Input} \textit{After}}
\end{figure}

\hspace{2em} Gambar 3.48 menunjukkan tampilan halaman \textit{Event Input} setelah dilakukan \textit{revamp} pada bagian \textit{form input}. Pada halaman ini ditambahkan \textit{form field Description} untuk memberikan catatan terkait \textit{event}, serta fitur \textit{Upload Media} untuk mengunggah \textit{file} yang diperlukan, seperti \textit{file} PPT untuk keperluan \textit{meeting}.

\begin{figure}
    \centering
    \fbox{\includegraphics[width=1\linewidth]{assets/pics/event input holiday.png}}
    \caption{Tampilan halaman \textit{Event Input type Holiday}}
\end{figure}

\hspace{2em} Gambar 3.49 menunjukkan tampilan halaman \textit{Event Input} dengan tipe holiday. Pada bagian \textit{form input} untuk tipe ini, hanya terdapat \textit{form field Event Name} untuk memberikan nama kegiatan libur, tanggal pelaksanaan yang ditetapkan, serta \textit{form field opsional} seperti \textit{description} dan upload document sebagai kelengkapan informasi terkait hari libur tersebut.

\section{Kendala dan Solusi yang Ditemukan}

\hspace{2em} Beberapa kendala yang ditemui selama melaksanakan kegiatan magang di PT Visi Karya Nusantara adalah sebagai berikut:

\begin{enumerate}
    \item Terjadinya ketidaksesuaian komunikasi antara tim \textit{backend} dan \textit{frontend} pada tahap awal magang, yang menyebabkan terjadinya miskomunikasi dan memerlukan penyesuaian tambahan saat proses integrasi.
    
    \item Kesulitan dalam menentukan alur fitur atau modul yang akan dikembangkan, sehingga membutuhkan waktu lebih untuk melakukan analisis serta riset yang mendalam.
\end{enumerate}

\hspace{2em} Adapun solusi yang diterapkan untuk mengatasi kendala tersebut adalah sebagai berikut:

\begin{enumerate}
    \item Meningkatkan kualitas komunikasi antar tim \textit{backend} dan \textit{frontend} guna memastikan kesesuaian saat proses integrasi.
    
    \item  Memperdalam pemahaman terhadap konsep fitur yang dikembangkan serta melakukan riset lebih lanjut terkait kebutuhan dan ekspektasi pengguna terhadap fitur atau modul tersebut.
\end{enumerate}




%---------------------------------------------------------------
\chapter{\babEmpat}
%---------------------------------------------------------------

%---------------------------------------------------------------
\section{Simpulan}
%---------------------------------------------------------------
\hspace{2em} Pengembangan \textit{website Human Resource Information System} (HRIS) pada PT Visi Karya Nusantara bertujuan untuk mempermudah tim \textit{Human Resource} dalam melakukan koordinasi, pengelolaan data, serta monitoring aktivitas karyawan secara menyeluruh. Seluruh proses yang sebelumnya dilakukan secara manual kini telah terintegrasi dalam satu sistem, sehingga alur kerja menjadi lebih efisien, terstruktur, dan mudah diakses kapan pun dibutuhkan. \textit{Website} ini telah sepenuhnya dikembangkan dan dipublikasikan dalam mode \textit{production}, sehingga dapat langsung digunakan oleh klien untuk mendukung berbagai kebutuhan operasional seperti koordinasi internal, pemantauan kehadiran, pengajuan dokumen, hingga pengelolaan kontrak dan jadwal kegiatan.

\hspace{2em} Dalam proses pengembangannya, HRIS dibangun menggunakan framework React TypeScript pada sisi \textit{frontend} guna memastikan tampilan antarmuka yang responsif, interaktif, dan mudah digunakan. Sementara itu, sisi \textit{backend} dikembangkan menggunakan ExpressJS untuk menyediakan layanan API yang stabil, cepat, dan terstruktur. Hasil akhir dari pengembangan HRIS mencakup beberapa modul utama, yaitu \textit{Contract, Payroll System, Event,} serta \textit{Dashboard} yang di dalamnya terdapat fitur \textit{Attendance}. Setiap modul dirancang untuk saling terintegrasi sehingga mampu mendukung keseluruhan proses bisnis perusahaan secara efektif dan berkelanjutan.
%---------------------------------------------------------------
\section{Saran}
%---------------------------------------------------------------

\hspace{2em} Selama pelaksanaan kegiatan magang di PT Visi Karya Nusantara, terdapat beberapa saran yang dapat diberikan untuk mendukung pengembangan sistem kedepannya. Adapun saran-saran tersebut adalah sebagai berikut:

\begin{enumerate}
    \item Penambahan fitur kasbon untuk mempermudah tim \textit{Human Resource} dalam melakukan pendataan karyawan yang mengajukan pinjaman kepada perusahaan, sekaligus menjadi bukti resmi terkait pengajuan pinjaman tersebut.
    \pagebreak

    \item Melakukan integrasi jadwal \textit{holiday event} pada fitur \textit{attendance} dan pengajuan cuti, sehingga karyawan tidak tercatat sebagai \textit{missing in action} apabila tidak melakukan absensi pada hari yang telah ditetapkan sebagai hari libur. Selain itu, pengajuan cuti juga tidak akan berkurang apabila tanggal tersebut termasuk dalam hari libur.

    \item Menambahkan sistem penggajian pada modul \textit{Contract} yang mendukung perhitungan harian, mingguan, bulanan, maupun tahunan, serta dilengkapi dengan perhitungan gaji pokok secara otomatis berdasarkan interval yang dipilih.
\end{enumerate}

%
% Daftar Pustaka
%
\singlespacing

% \vspace{-1cm}

\bibliographystyle{IEEEtran}
\bibliography{referencess}



\onehalfspacing

%
% Lampiran 
%

%-----------------------------------------------------------------------------%
% \addChapter{DAFTAR LAMPIRAN}
% \chapter*{Daftar Lampiran}
% %-----------------------------------------------------------------------------%
\newpage
\appendix
% \appendices{Lampiran 1. Form Bimbingan} 
\renewcommand{\thetable}{A\arabic{table}}
\renewcommand{\thefigure}{A\arabic{figure}}
\renewcommand{\thelstlisting}{A\arabic{lstlisting}}


\appendices{MBKM-01 Cover Letter MBKM Internship Track 2}
% \section*{Lampiran 1. MBKM-01 Cover Letter MBKM Internship Track 1}
\begin{center}
    \frame{\includegraphics[width=\textwidth,page=1]{assets/pdf/Cover latter - amanda citra dewanti.pdf}}
\end{center}
\clearpage


\appendices{MBKM-02 MBKM Internship Track 2 Card}
% \section*{Lampiran 2. MBKM-02 MBKM Internship Track 1 Card}
\begin{center}
    \frame{\includegraphics[width=\textwidth,page=1]{assets/pdf/mbkm magang 2 .pdf}}
\end{center}
\clearpage


\appendices{MBKM-03 Daily Task - Internship Track 2}
% \section*{Lampiran 3. MBKM-03 Daily Task - Internship Track 1}
\begin{center}
    \frame{\includegraphics[width=\textwidth]{assets/pdf/daily task magang 2.pdf}}
\end{center}
\clearpage


\appendices{MBKM-04 Verification Form of Internship Report MBKM Internship Track 2}
% \section*{Lampiran 4. MBKM-04 Verification Form of Internship Report MBKM Internship Track 1}
\begin{center}
    \frame{\includegraphics[width=\textwidth,page=1]{assets/pdf/verification magang 2.pdf}}
\end{center}
\clearpage

\appendices{PRO-STEP-05 Surat Penerimaan}
 % \section*{Lampiran 4. MBKM-04 Verification Form of Internship Report MBKM Internship Track 1}
\begin{center}
    \frame{\includegraphics[width=\textwidth,page=1]{assets/pdf/Offering Letter - Amanda Citra Dewanti.pdf}}
\end{center}
\clearpage


\appendices{Form Bimbingan} 
% \section*{Lampiran 5. Form Bimbingan}
\begin{center}
    \frame{\includegraphics[width=\textwidth,page=1]{assets/pdf/00000066344_Counseling_AllMeeting (1).pdf}}
\end{center}
\clearpage

\appendices{Hasil Pengecekan Similarity Turnitin} 
% \section*{Lampiran 6. Hasil Pengecekan Turnitin}
% Include all pages from the multi-page Turnitin PDF reliably
\begin{center}
    \frame{\includegraphics[width=\textwidth,pages=1-3,fitpaper=true,pagecommand={}]{assets/pdf/turnitin.pdf}}
\end{center}
\clearpage

\appendices{Formulir Penggunaan Perangkat Kecerdasan Artifisial (AI)}
\noindent
\begin{tabular}{lcl}
   Nama  &:& \penulis \\
   NIM  &:& \nim \\
   Email &:& Email \\
   Program Studi &:& \program \\
%   Fakultas &:& \fakultas
\end{tabular}

\noindent
% \usepackage{array}
\setlength{\LTleft}{0pt}
\setlength{\LTright}{0pt}
\setlength{\extrarowheight}{2pt}

{\footnotesize
\begin{longtable}{
|p{0.4cm}
|p{2cm}
|p{3cm}
|m{4cm}
|p{1.5cm}
|m{3cm}|}   % m{} for image
\hline

\centering\textbf{No} &
\centering\textbf{Nama Tool} &
\centering\textbf{Alamat Web/Url} &
\centering\textbf{Prompt} &
\centering\textbf{Tanggal Akses} &
\centering\textbf{Media Output} \\
\hline
\endfirsthead

\hline
\centering\textbf{No} &
\centering\textbf{Nama Tool} &
\centering\textbf{Alamat Web/Url} &
\centering\textbf{Prompt} &
\centering\textbf{Tanggal Akses} &
\centering\textbf{Media Output} \\
\hline
\endhead

\centering 1 &
\centering ChatGpt &
\centering https://chatgpt.com/ &
rapihkan dan parafrase tipis-tipis &
\centering 3 December 2025 &
\centering\includegraphics[width=\linewidth]{assets/pics/promp 1.png} \\ \hline

\centering 2 &
\centering ChatGpt &
\centering https://chatgpt.com/ &
sitasikan dalam latex &
\centering  25 september 2025 &
\centering\includegraphics[width=\linewidth]{assets/pics/promp 2.png} \\ \hline

\centering 3 &
\centering ChatGpt &
\centering https://chatgpt.com/ &
ini kenapa latex saya error ga bisa di complie &
\centering 10 December 2025 &
\centering\includegraphics[width=\linewidth]{assets/pics/promp 3.png} \\ \hline

\end{longtable}
}







\end{document}